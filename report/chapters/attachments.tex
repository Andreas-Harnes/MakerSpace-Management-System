\chapter{Attachments}

\subsection{Heuristic evaluation results}

\begin{enumerate}
    \item \textbf{How are things categorized?}
    
\textbf{Group member:} 
The first thing that you see when going to finn.no is lots of elements that is categories. It is up to 14 different categories, and each of the elements is visualized with symbols, alongside with a category name. That is an good combination to help the user to choose the right category when they are going to look after stuff to buy. With only having symbols, it would lead up to misunderstanding, since some symbols can be too old for someone (as an old telephone for the young people), or people can have different interpretations about the symbolse. It is a good way to categorize. 

\textbf{Assignment provider:}
Things are categorized by primary type like “Eiendom”, “Bil”, “Torget” and then categories by sub category under the primary category. Some categories have rather deep category trees.

    \item \textbf{How does search work?}

\textbf{Group member:} 
The search is also something you find on the first site. In the search it have been placed a placeholder/label with a few examples to help the user to know how or what they can search for. 

When searching for example “TV”, you will get how many hits you get for each categories. This is a good thing, since if you know that you are searching for a “TV” as in a television, then you know that you need to choose the category where you can find television, and not the category about apartment. 

\textbf{Assignment provider:}
Search works very well. When in a category the search only searches in that category, which is a rather powerful feature of the system. On the main page however, search works on all items on the site.

    \item \textbf{How many things can you see on a page before switching?}

\textbf{Group member:} 
When you are looking on all the hits you got by searching television, you can see up to 51 elements before you can switch to the next site. The buy elements is organized like it is tree elements side by side, and then a new line with tree new elements. 

\textbf{Assignment provider:}
50 items in the electronics category.

    \item \textbf{Can you create your own user?}

\textbf{Group member:} 
Yes it is possible to create your own user on finn.no. In the upper right corner of the front page, it says “My FINN”. When clicking on that you will be taken to a page where you got the opportunity to logg inn or make a user. When clicking on “make a user” you will need to fill in email and password. 

\textbf{Assignment provider:}
Yes. It’s fairly simple, but you do have to verify with BankID to authenticate your account.

    \item \textbf{Is it easy to find contact information?}

\textbf{Group member:} 
To find contact information to FINN, you need to scroll down to the footer. It is written with small texts, and can for someone be hard to find at first. It should be placed on top or more visible so the user don't need to look for it. But on the other hand the FINN’s contact information is not the most important thing, it is to find the contact information to the one you are gonna buy a item from.

Without clicking the advertisement, the only information you get is where the item is and if the are a company or private seller. When you click on the advertisement you will on the right side get the name of the seller and right below it you will find a button you can click that give you opportunity to send a message/email to the seller. With other words it is easy to find.

\textbf{Assignment provider:}
Yes, it’s easy to find contact information for both Finn.no and individual sellers.
	
    \item \textbf{How does navigation work?}

\textbf{Group member:} 
When doing navigation on the main page you go through the most important things first. You start with search. After that you will navigate through all of the category that exist and then through some popular advertisement. At last you navigate through the footer where social media and contact information is. 

When you are inside of an advertisement you navigate first through the picture and information about the item. After that you going through contact information about the seller. It feel neat and orientated to navigate through FINN.

\textbf{Assignment provider:}
As Finn.no is a site for selling and buying stuff navigation primarily works on item category. As a sales site the navigation is pretty effective.

    \item \textbf{Can you store things you are interested in?}

\textbf{Group member:} 
FINN offers a element that let you store advertisement that you are interesting or want to get information about it if it has been updated or of it still for sale. Before clicking the advertisement you will see on the ad that its a heart symbol. When clicking that you will store it in your favorit. You can also add it to favorit when you have clicked on it, it will show a element below the picture that says “Add to favorit”

\textbf{Assignment provider:}
Yes. If you have an account, you just need to push the heart or “Legg til favoritt”.
	
    \item \textbf{What does the buy item contain?}

\textbf{Group member:} 
On the item it shows picture(s) of the product, follow by the title for the advertisement. For som product it also shows specific things and equipment and below that you will see the price for it. After that you can read the description about the product.

\textbf{Assignment provider:}
There are no “Buy item” button on the site. As the site is an action site you need to message the seller to be able to buy the item. Professional sellers that use Finn.no to showcase their products have a “Buy in online store” button where you can buy the items. To buy stuff from private sellers you need to send the seller a message and then buy the item directly.
\end{enumerate}

\textbf{Summary}

Since both of the evaluators is having two different expertise it was expected to get some different result of the evaluating. That is good for seeing more than one perspective of looking into FINN’s website. This is what that have been found by following the eight heuristics: 

\textbf{1.  How are things categorized?}
Things are categorized by primary type and then by sub category. They also use symbols/icons to represent the categories. 

\textbf{2.  How does search work?}
The search is placed good where it is very simple
to find it. When using it, it will in a category only searches in that category, which is a powerful feature of the system. On the main page it will search for all of the items. 

\textbf{3.  How many things can you see on a page before switching?}
50 element is possible to see. 

\textbf{4.  Can you create your own user?}
Yes it is possible and very simple. When creating a user one need to use BankID to authenticate your account.

\textbf{5.  Is it easy to find contact information?}
Yes, it is simple to find contact information for the seller, but a little tricky to find contact information for FINN. It has not been prioritized since it will be some of the last thing one will get by navigating through the main page. 

\textbf{6.  How does navigation work?}
The navigation go through the most important first, with that it means sals items and it works alright. 

\textbf{7.  Can you store things you are interested in?}
If one is having a account, then it is possible to store things of interest. It is a button that can be clicked on for storing. 

\textbf{8.  What does the buy item contain?}
It contain picture, title, description and other important information that needed before buying. For private sellers one need to take direct contact/message the seller to buy the item. For professional sellers it will be displayed a “Buy in online store” button.

\subsection{Test plan for usability testing}
\begin{enumerate}
  \item \textbf{Purpose:} This is where you describe the purpose of what you will find with this usability test.
  \item \textbf{Problem statement:} Here you need to make some kind of list of problems that you want to be answered through the test.
  \item \textbf{User profile:} This is where you need to describe what kind of users that is supposed to take the test. What age are they? What background information do they have on what they will be testing? It is important to describe your user to see if that will affect the results of the test.
  \item \textbf{Methodology:} This is more of an concrete way to describe how the test should be carried out. Often it is four steps that should be followed. Those four steps are: 
  \begin{enumerate}
  \item \textbf{Participant greeting and background questionnaire:} This is where you can greet the participants and make them feel welcome. Do a background questionnaire and make sure that an anonymity agreement is signed.
  
  \item \textbf{Orientation:} The participant will get a short, verbal and scripted introduction and orientation about the test.
  The purpose of the test will be explained and what is expected from them as a participant. The participant will also be informed that they will be observed, videotaped or audio taped.
  
  \item \textbf{Performance test:} The performance test consist of a series of tasks and/or scenarios that the participant will be asked to do while being observed. The observer will write down any of the participants behavior, comment and any other unusual circumstances that might affect the result of the test.       
  
  \item \textbf{Participant debriefing:} After the test is done, it's time for a debriefing with the participant. It should follow these steps:
  
  - Filling out some kind of a brief performance questionnaire of the usability of the system. 
  
  - Let the participant give an overall comment of her or his performance of the test.
  
  - Let the participant tell about errors or problems that occurred during the test.
  
  \end{enumerate}
  \item \textbf{Task list:} This will contain a list of tasks that the monitor of the test want the participants to do. This is in order for the monitor to be able to test the things that he or she wants to test.
  
  \item \textbf{Test environment and equipment requirements:} This one need to describe the environment that the participant will be in. Is the environment quiet and calm or is it a lot of disturbance around that can disturb and affect the test result? That is important to document.
  
  Alongside with a description of the environment, it can be necessary to write down what kind of equipment that will be needed to complete the test. An example can be if there is a need for a pen, paper or maybe a phone. 
  
  \item \textbf{Test monitor role:} This is a description of what kind of role the test monitor will have during the test. Is the monitor going to be closed or open. By that it means will the monitor be closed as in the participant wont get any help if they ask, or if the monitor will be open where the monitor can answer questions and give some leads.
  
  It will need to be describe what the test monitor will be looking for. If the test is recorded, are the monitor still gonna write notes during the test, or is he or she just gonna look at the recording afterwards?  
  
  \item \textbf{Evaluation measure:} This is what kind of information or data that will be collected through the testing. It can be how long the participant use to complete the task, how many errors that occurs or other things that is needs to be collected.

  \item \textbf{Test report contents and presentation:} This is what the usability test consist of, and what the reader can find. It mainly consists of the test plan, the result and findings found in the test.
\end{enumerate}
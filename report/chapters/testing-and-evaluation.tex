\chapter{Results from data gathering and evaluation}
\todo{endre overskrift og beskrivelse tekst}
In this chapter it can be read about the evaluation and testing method, results from gathering data through a questionnaire and a interview that has been done for the project. The first part is about the results from gathering data about different MakerSpaces around Norway. The results comes from an interview and a questionnaire. The second part that can be read about is the heuristic evaluation where the focus has been looking for some design ideas that can be implemented and where it has also provided the basis for a wireframe. In the third part it can be read about the usability test that has been done. In the test it can be seen how well the users could use the system, what errors that emerged and how well the functionality works so far. 

\section{Results from analyzing different MakerSpaces}
 A questionnaire has been sent out to as many MakerSpace's as possible. A list of the recipients can be found as an attachment.    

\subsection{MakerSpace Inspiria interview}

This section is based on an interview the group conducted, as such the necessary sources are added as an attachment. The source is a voice recording of the interview transcribed.It can be found as an attachment.  

The first MakerSpace that was interviewed was MakerSpace Inspiria, located in Sarpsborg. It is part of Inspiria Science Centre \cite{Inspiria-SC}, a building full of science for people to enjoy and test out. It is mainly targeted for kids, but the MakerSpace is open for everyone when the centre is open. MakerSpace Inspiria is very much alike MakerSpace Halden. It's closely resembles each other and has a lot of the same equipment; 3D printers, laser cutters, soldering irons etc. 
And it is here the group conducted its first interview, at MakerSpace Inspiria. The interviewee answered all the questions best possible and the group got a lot of interesting information. There was no system in place for keeping tabs on the inventory, digital or other. The employees were the ones that kept control over the inventory personally, the employees shared responsibility of overseeing the inventory and ordering new parts and equipment when needed. But admitted to not having full control over everything. Components and tools are being kept in lockers and has a box i.e where similar tools or components are grouped and put together. There is however no special categorizing of the equipment, but everything has a predetermined place. 

As far as lending out equipment, MakerSpace Inspiria has no system in place for it and they do not lend equipment out. Exceptions could be made, depending on the person asking to loan something or the equipment in question. If an exception is made, the employee that loans something out is responsible for the item and its condition. If something is broken or gets broken, MakerSpace Inspiria has a system for reporting broken or faulty equipment. This system is used for the entire building (Inspiria Science Center). The employees report on the specific equipment that is broken and it is filed for review. The review consist of looking at the budget and the broken equipment to see if it is possible to order a replacement or if it is to expensive at the moment. This depends on what kind of equipment is broken. If it is an expensive 3D printer there might not be enough money in the budget to buy a new one e.g However if it is something smaller and not very expensive the employees doesn't report it broken and just orders a new one. 

When asked if a digital system for inventory keeping was something MakerSpace Inspiria could potentially want, the interviewee positively agreed. The interviewee already had an idea for a potential digital system. A simple bar-code and a scanner, zero thinking. Everything has its own ID and it gets scanned into a system. Everything about it has to be simple and have as few button presses as possible. This was from the interviewees own experience with people and IT, or more about people who doesn't know a lot about IT and would be using mentioned system. Other than that the interviewee didn't have any specific ideas on how the system should work or look. But the project group got some very nice ideas and pointers for the User Interface (UI) and User Experience (UX). Keep it nice and simple, not very complicated which is something the project group had already thought of. A bar-code was discussed early on in the project as a possible solution for both lending out equipment and inventory keeping. 

\subsection{Results from the questionnaire}

After an interview and getting 20 questionnaire responses from different MakerSpace's. The group has read through the responses and analyzed them. However, since all the answers are anonymous the group doesn't know which MakerSpace's that responded. The equipment each respondent have was varied, everything from digital tools and equipment to more heavy tools like CNC machines, laser cutter, metal saws, vinyl cutters, acids for creating circuit boards and more. Basically every tool for the job to let everyone, children and adults be as creative as possible and give them the space they need for it. 

For any workplace that have tools, it is important to know where everything is. This makes it easier to find what you are looking for. Out of 20 respondents the categorizing and placement of tools went from having no idea or no system, to full control with a website and database for every tool and component. The most common sorting was after the size and type of tool and what department it belonged to (Electronics, Mechanical i.e). Here is a short list of different methods of storing and categorizing tools from MakerSpace's:

\begin{itemize}
    \item "Big things goes in lockers, small things goes on the wall."
    \item "Categorized after which department it belongs to, i.e electronics, metal, wood etc."
    \item "Tools are not categorized and all components counted by stocktaking and saved in a database with potential of resale through own proprietary developed sale system."  
    \item "Every component is sorted after the category i.e transistors, IC, sensors and tools are sorted after usage area and size. There is a designated room for the biggest/baddest/noisiest tools." 
    \item "I have no clue"
    \item "Drawers and shelves etc. On our website you can search i.e for an electronic component and get an overview of what we have and the exact shelf it is located on, but also if the component is available in stock. Sale price is also listed. Heavy machinery like welding gear, wood grinder belongs to a room we call heavy machinery.
    Most of the electronic components are sorted into shelves with a searchable position from our website. However quantity is not always updated, you will need to check yourself if the component you are looking for is available. If we are informed that we have run empty of a component, it will take usually 1-2 weeks before we have bought in more. We sell Arduino, Raspberry pi and touch monitors to customers. We also have two new room where we have a laser cutter and other stuff but the room Isn't fully decorated yet."  
\end{itemize}

Tools, equipment and components can be broken. How the broken item is reported broken and handled afterwards was one of the questions asked. How this is handled with or without a digital system could be important information for how the application is developed and how items in the system should be handled(function as removable by unique id or based on quantity i.e). 
None of the respondents had any digital system for reporting damaged or destroyed tools and components. If anything was broken it was either reported to a superuser or employee by the person that broke or found something broken. Some of the tools and components was considered consumable and just replaced. Certain equipment at one MakerSpace was prohibited from "ordinary users" from using and only certain members was allowed to operate them. Some examples of what equipment was prohibited is; 3D printers, CNC machines, laser cutter and welding tools. The prohibition of the equipment happened after someone broke a very expensive 3D printer. 
Some of the respondents had a form that needed to be filled out and also used Slack internally for the employees to communicate. Slack is a cloud based system used by teams(development, project management i.e) to communicate with each other\cite{Slack_Software}. This includes chat room, private groups and direct messaging. The very convenient about Slack is that all content inside slack is searchable. Including files, conversations and people\cite{what_is_slack}.

60\% said people are allowed to borrow tools and equipment home. The people that could borrow things was restricted to members of the MakerSpace, everyone with a contract, members but something needs to be pawned as collateral i.e drivers license, members or students, students. Out of the people that could borrow tools and equipment, there were some restrictions to what was allowed to borrow.    

\begin{itemize}
    \item Some items had a chip that marked them as loan able.
    \item Only small tools that won't be missed or there is more than one of them.
    \item Only equipment that there was more than one of, but collateral had to be left behind. Like a drivers license,keys or bankcard.
    \item No limits on what members can borrow
    \item Some equipment needed training and membership.
    \item Items are loaned out based on how well they are known to the employees but expensive or advanced equipment is something that is rarely loaned out to members. Unless they have a really good reason or makes a written deal.
\end{itemize}

The way every item was kept track of when it was lent out. Was either by utilizing a library's loaning system, written down in a log, orally or no overview at all. Of the ones who used a written system, one never really enforced the use of the log but the system worked because they hostages in the form of driver licenses. phones, keys and most owners of these were very eager to get their things back. Another used Google Sheet with the name of the person that is loaning something, what they are loaning, how they can be contacted, who lent it to the person(employee etc) and date for planned delivery.
Two of the MakerSpace's that used a log(paper) reported that they have a digital system in development where one said the system would incorporate pawning of a valuable item as collateral in the system. 
If the borrowed items didn't get returned a number of things would happen to the people that borrowed something.(Nevn halden makerspace) A compensation claim would be sent if the person couldn't be contacted or could be contacted but didn't return the item regardless. Most of the time, contacting the person would resolve the problem peacefully with the item returned. The MakerSpace's that took hostages would keep the valuable items left as collateral, one reported sending the collateral for destruction if the loaned item wasn't returned. Another MakerSpace reported that if an item wasn't returned they would yell at them, angrily. 

The 20 responses that the group got from the form reported that 55\% did not have a inventory system to keep track of their tools, equipment and components. 45\% reported they had a system. Out of the 45\% that had a inventory system, 66.7\% said the system was digital(6 respondents) and 33.3\% said it was not digital. If the system wasn't digital most of the other system was either a written one, like a list, marked shelves or orally where the employees have control. Most of the access to the inventory system was limited to one person or admins, but everyone had access to basic usage like search options to see what and how much of something was in stock for the digital systems. 
The detail of the systems the different MakerSpace's had was varied in detail but similar how it all was organized. One MakerSpace had every components in one permanent place with quantity of the item. No info on how the tools where being kept in the system. Another had every component in small marked drawers and tools in shelves, on the wall or in other, bigger drawers. Some of the digital systems used was either developed on their own, or used Google-spreadsheet. In the Google-spreadsheet each item was detailed as well as possible, but consumable components are not part of the inventory system. The most detailed digital inventory system is quite extensive and well developed. It had a very detailed system for storing components looking like this:

\begin{itemize}
    \item Name 
    \item Description
    \item Producer
    \item The type of component (10 Ohm, 100 Ohm i.e)
    \item Quantity
    \item Price
    \item Shelf placement (four null-indexed values to find it) 
    \item Section
    \item Row and column
    \item Depth (Some boxes and drawers might have several components in it but at different depth)
\end{itemize}
This system was coupled up with a cashless system where payments was done through "iZettle"\cite{iZettle_Financial_Products}. IZettle is a Swedish financial company and offers some finanical products like payments through their app. IZettle let's smaller companies accept card payments\cite{What_is_iZettle}. For payments students and other users of the MakerSpace could use their student id card or an "RFID" chip to make payments for components. RFID is a technology where digital data can be encoded in RFID tags or smart labels. And the data can be read by capturing radio waves. RFID is an acronym for "radio-frequency identification"\cite{What_is_RFID}. However the MakerSpace did not have any lists or a system in place over tools or equipment they loan out. Most of their users use the workshop and it's tools and equipment there. 

New items in the system was registered either by the one responsible for buying the components or delegated the task to someone else. For deleting or removing items in the system. It was done by the system updating itself when an item was removed, sold or by stocktaking at some point. Other alternatives was switching the item for a new one based on shelf location or removing the item from the system entirely if it was broken.
For those that had a system in place for inventory, they got a follow up question if there was something that could be different or better with the system in place. Some of the note worthy changes was better logging of items in the system with an alert when stock is low. Better digital logging for loaning of tools, easier to use/search through, better logging of history and some wanted a web-shop function for sale of consumable items. 

There were a set of MakerSpaces that didn't have a inventory system. And most of them had control over items by memory and that employees or volunteers also keeps an overview over what they have and where it is. A lot of the control is based on trust and respect over the workshop and that no one steals and keeps everything tidy. Of the eleven responses that didn't have a inventory system, 54.5\% said they did not want a digital system. And if they were to have a digital system the functions that was most wanted from a system was; a loan function, overview/list function, barcode for each item or RFID, a experience log for logging how to use equipment, settings, adjustments, mistakes and tricks. A function that could "tag" equipment that is broken combined with a detail history log for whether it has been broken before and how it was fixed. The option of adding video, pdf or a picture to what is broken or has been broken was a potential function a MakerSpace wanted. A wish list that employees could use to add wanted equipment and a shift log/report for employees where they can note needs, problems and other issues. 
When asked if there was something they wished was different by how they organize and categorize their equipment and tools compared to now. The answer was; better overview, more space, an inventory system,  a system for more control over items and better marked equipment to be able to have a better overview and make it automatic. So there is a lot of room for improvement and changes based on the answers from the questionnaire.

\subsection{Conclusion of the interview and questionnaire}
After one interview and several responses on the questionnaire the project group has gathered a lot of useful information that can be used to improve the project. And with that information the project group has learned a lot about MakerSpace's around Norway. One thing the project group learned was that many did not have any system in place for inventory keeping or for lending out equipment, digital or other. However the employees or volunteers would be the ones that either memorized or had personally control over where think was located. If something was loaned out it was done by either taking a "hostage", this was the case with digital systems as well. Or it was based on respect/the employees knowing who they lend something to, but many surprisingly used some form of paper log or just an oral deal. Which is similar to how MakerSpace HiØ operates with today and the control over equipment and tools was similar between MakerSpace HiØ, MakerSpace Inspiria and from MakerSpace responses. Where the employees and volunteers have control and items are placed in general places that isn't assigned but "that's where they go". 
For the digital systems there was six confirmed systems in place with many of the features MakerSpace HiØ needs and could want. Which means many MakerSpace's feels a bit disorganized at times and want better control. Of all the responses the project group got five that said they wanted a digital system and was either planning or developing one at the time of answering. So developing a digital system is definitely useful for many MakerSpace's and makes this project something that could be potentially used at other MakerSpace's when fully developed. Many ideas the project group had like potentially using RFID and student id's for registering users and using it to loan equipment. Is also something other MakerSpace's and the project group has thought about. 

So from the interview and questionnaire the project group found out that some MakerSpace's are in the same place as MakerSpace HiØ and seeking better systems to keep control over their workshop. And some MakerSpace's are where MakerSpace HiØ wants to be in the future. The responses have provided important information about vital functions a digital system should and could have. As well as shown ideas to how a digital system could be done. This information will be an imoportant part for the project and for further development of the project. As it could lead to potential cooperation between MakerSpace's for developing a joint system. Since many sees the need and usefulness of a digital system for management and equipment lending.

\section{Heuristic evaluation}
The purposes with this heuristic evaluation is that the group is going to look at a another website that share some of the similarity as the group think the system in development will have. For this evaluation the group alongside with the assignment provider has decided that they are going to look into finn.no website. The test itself will be performed by one of the group members and by the assignment provider.

FINN is a website where users can exchange and sell stuff that they have at home. The group think that by looking at FINN, it will give some ideas to how the inventory system will look like, categorizing things and what kind of layout the website will have. FINN have also been nominated for best interactive design,
and considered to have good universal design.\cite{finn_nominert}
That makes it a good choice to look into, since universal design is also something the group need to implement in the development of the system. 

As mentioned in the section about method, when doing a heuristic evaluation there needs to be some rules set in place. These rules needs to be followed throughout the evaluation. The rules should contain what things that will be look for when doing the evaluation. The heuristics that have been agreed to are:

\subsection{Heuristics}
\begin{enumerate}
  \item How are things categorized?
  \item How does search work?
  \item How many things can you see on a page before switching? 
  \item Can you create your own user?
  \item Is it easy to find contact information?
  \item How does navigation work?
  \item Can you store things you are interested in?
  \item What does the buy item contain?
 \end{enumerate}

\textit{See attachment for results.}
%Legger til vedlegg av selve test dokumentet/resultater%

\subsection{Summary}
Since both of the evaluators is having two different expertise it was expected to get some different result of the evaluating. That is good for seeing more than one perspective of looking into FINN’s website. This is what that have been found by following the eight heuristics: 
\begin{enumerate}
  \item \textbf{How are things categorized?} Things are categorized by primary type and then by sub category. They also use symbols/icons to represent the categories. 
  \item \textbf{How does search work?} The search is placed good where it is very simple
to find it. When using it, it will in a category only searches in that category, which is a powerful feature of the system. On the main page it will search for all of the items. 

  \item \textbf{How many things can you see on a page before switching?} 50 element is possible to see. 
  \item \textbf{Can you create your own user?} Yes it is possible and very simple. When creating a user one need to use BankID to authenticate your account.
  \item \textbf{Is it easy to find contact information?} Yes, it is simple to find contact information for the seller, but a little tricky to find contact information for FINN. It has not been prioritized since it will be some of the last thing one will get by navigating through the main page. 
  \item \textbf{How does navigation work?} The navigation go through the most important first, with that it means sals items and it works alright. 
  \item \textbf{Can you store things you are interested in?} If one is having a account, then it is possible to store things of interest. It is a button that can be clicked on for storing. 
  \item \textbf{What does the buy item contain?} It contain picture, title, description and other important information that needed before buying. For private sellers one need to take direct contact/message the seller to buy the item. For professional sellers it will be displayed a “Buy in online store” button.
 \end{enumerate}

\section{Usability testing} 
Since the group decided to use heuristic evaluation to try to look for some good ideas about design \cite{heuristic-evaluation} and how items have been 
categorized, the usability test will now be used to test the functionality for the system they are building. This to find out if the things the group has developed is working, or if there are bugs and perhaps some additional components that can be added.

The group will do this test on those who work on MakerSpace to see if they wish something was done different or if they feel something is missing to meet the needs for this system. The group will also find other students from other educations than Information Technology (IT), to see if those students can find different errors. Since this system is not only for the students that studies IT, but the whole school, it is important to test different students from different studies to make it fit everyone.

At this time the group already now that the system are facing some issues. Things like the user will not know the rules of how making a password and they will not know if the registration is done correctly, therefore the monitor of the test will give some help before going to next task. The monitor i suppose to be passive, but without any help it can be hard to test the actually functionality of this system. Other things is that items is at the moment before testing not displaying so it might be hard for the user to understand that they got right when looking for items.

Below you can read about how the group will conduct the usability test and how the result for the tests are.

\subsection{Test plan}
\textbf{Purpose:} The purpose of this test is to detect errors and test the functionality of the system.

\textbf{Problem statement:} The problem that will be tested will be if the participant can register a new user, alongside if the participant can log in with the user he/she created. Another thing that will be tested is if the participant can navigat back to the home page/item page. More details of the exact tasks are found in "Task list" bellow. 

\textbf{Test roles:} In this test it will be two different roles. One moderator and one observer. The moderator will be the one who heads the test. With that it means the moderator will read out the problems/task that the participant will be doing and also give out follow-up questions. The moderator role will be semi-open. That means the moderator will not help the participant with the test, but can can answer questions that the participant have, if there are some unclear things about the test task.

The observer role will be completely passive. The only thing the observer will do is to note down some data that is gonna be collected by this test. It can be a certain behavior that the participant have through the test, like wrong "click" or how long it takes for the participant to complete the test. Other behavior which can affect the test result will also be noted by the observer. Since it will be used a Thinking Aloud method, the observer will need to note down what the participant is thinking when doing the test. 

\textbf{User profile:} The user that will be tested is two persons that work at MakerSpace and the group will also find one other student on the school for example, a teacher student.

\textbf{Methodology:} 
The group has decided that they will divide the test in this four steps:
\begin{enumerate}
    \item \textbf{Background questionnaire:} This is where the moderator for the test collect background information about the participant before starting the test. This is going to be be gender, age and what education they take, or whether it is a student or a teacher. The moderator also make sure that the anonymity agreement will be signed before starting \textit{(see appendix "Use test template")}.
    
    \item \textbf{Orientation:} The moderator will read and give out different problems/tasks that the participant will perform.
    
    \item \textbf{Performance test:} When the participant is preforming the test, then the observer will take notes of different things that has been agreed to collect.
    
    \item \textbf{Participant debriefing:} After the test is done, the moderator will ask some follow-up questions. This is to find out how the participant felt the test went or if they have anything else they want to add before finishing the test.
\end{enumerate}
\textbf{Task list:}
\begin{table}[h]


\begin{tabular}{lllll}
 \textbf{Nr.} & \textbf{Task} & \textbf{Claims} & \textbf{
Measure}  \\
\textbf{1.} & Create a new user &  Use drop down menu & Time, errors \\
\textbf{2.} & Log in with user & Use drop down menu & Time, errors  \\
\textbf{3.} & Get back to homepage & Use drop down menu & Time, errors 
\end{tabular}
\end{table}

\begin{table}[h]
\begin{tabular}{ll}
\textbf{Nr.} & \textbf{Description} \\
\textbf{1.} & You want to make a new user on MakerSpace Management System, how will you do it? \\
\textbf{2.} & You will now log in with the user you created, how will you do it? \\
\textbf{3.} & You are now logged in and will look at equipment that MakerSpace have, how will you do it? 
\end{tabular}
\end{table} 

\textbf{Test environment and equipment requirements:} The test will be carried out in a group room at the school. That will make sure that the participant will not be disturbed while conducting the test, as that the group room is a quiet place.  

Equipment that is needed is a computer with the latest functioning version of the MakerSpace Management System. A few pens is needed and also a physical copy of the test tasks, one to the moderator and one for each of the observers and participant. A paper where the moderator collect background information, and a something the observer can takes notes on.

\textbf{Evaluating measure:} The main thing that will be tested or be collect is if the system work. The group will check if the participant is able to register a new user and log in with it. The observer will take notes of all the errors the participant do. For example if the participant can't find the log in site or don't understand how or what he/she is going to fill out. The time it takes for the participant to complete each task will also be measured.

\subsection{Summary}
The age of the participant was between 20-30 years. Two of the participant is studying IT and one studying teacher. For all of the participant the whole test went very fast. For each task all of them spent between one minute and 30 seconds to about 40 seconds to complete the task. 

On task one there was none of the participant that successfully could create a valid user. That was something that was expected, since the group had not added use feedback. Otherwise did none of the participant have any issue to find and fill out the information. Although none had any big issue finding registrar, they still think it was a bit silly to have the link in the drop down, instead of a menu link. The group will consider taking out both log in and registrar from drop down to make it even easier for the user to find it.

Task two went well for all of the participant, after the monitor help the participant to make a valid user before moving on to this task. None had any big issues, only as in registrar that they wish it was in the header menu instead of drop down. Another feedback was that one of the participant wish the system had a enter function, so it is possible to just hit enter when done filling out instead of clicking the actually log in button.

The last task went well too, but although all of the participant found back tho the right page, they was facing some confusion. This was also something that was expected since the was no real item displayed on that page jet. That made some participant to start clicking on other stuff on the page, as available button and so on. Some feedback was that someone tried to click on the logo, cause they thought that would take them back to front page, something it did not. 

On the question of how hard the participant think it was to complete each task, they think the first task was a bit tricky and confusing, but after getting a valid user they had no problem completing the two other tasks. Other feedback was as mentioned more visible menu and enter functionality. The group will take all the feedback into and make the system better. 


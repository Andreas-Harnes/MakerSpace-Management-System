\chapter{Evaluating and testing}
In this chapter it can be read about the evaluating and testing method that is been done for the project. For the first part it can be read about the heuristic evaluation where the focus has been looking for some design idea that can be implemented and where it has also provided the basis for a wireframe. For the second part it could be read about the usability test that has been done. With that it can be seen how well the user could use the system, what errors that emerge and how well the functionality works so fare. 

\section{Heuristic evaluation}
The purposes with this heuristic evaluation is that the group is going to look at a another website that share some of the similarity as the group think the system in development will have. For this evaluation the group alongside with the assignment provider has decided that they are going to look into finn.no website. The test itself will be performed by one of the group members and by the assignment provider.

FINN is a website where users can exchange and sell stuff that they have at home. The group think that by looking at FINN, it will give some ideas to how the inventory system will look like, categorizing things and what kind of layout the website will have. FINN have also been nominated for best interactive design,
and considered to have good universal design.\cite{finn_nominert}
That makes it a good choice to look into, since universal design is also something the group need to implement in the development of the system. 

As mentioned in the section about method, when doing a heuristic evaluation there needs to be some rules set in place. These rules needs to be followed throughout the evaluation. The rules should contain what things that will be look for when doing the evaluation. The heuristics that have been agreed to are:

\subsection{Heuristics}
\begin{enumerate}
  \item How are things categorized?
  \item How does search work?
  \item How many things can you see on a page before switching? 
  \item Can you create your own user?
  \item Is it easy to find contact information?
  \item How does navigation work?
  \item Can you store things you are interested in?
  \item What does the buy item contain?
 \end{enumerate}

%Legger til vedlegg av selve test dokumentet/resultater%

\subsection{Summary}


\section{Usability testing}
Since the group decided to use heuristic evaluation to try to look for some good ideas about design and how items have been 
categorized, the usability test will now be used to test the functionality for the system they are building. This to find out if the things the group has developed is working, or if there are bugs and perhaps some additional components that can be added.

The group will do this test on those who work on MakerSpace to see if they wish something was done different or if they feel something is missing to meet the needs for this system. The group will also find other students from other educations than Information Technology (IT), to see if those students can find different errors. Since this system is not only for the students that studies IT, but the whole school, it is important to test different students from different studies to make it fit everyone.

At this time the group already now that the system are facing some issues. Things like the user will not know the rules of how making a password and they will not know if the registration is done correctly, therefore the monitor of the test will give some help before going to next task. The monitor i suppose to be passive, but without any help it can be hard to test the actually functionality of this system. Other things is that items is at the moment before testing not displaying so it might be hard for the user to understand that they got right when looking for items.

Below you can read about how the group will conduct the usability test and how the result for the tests are.

\subsection{Test plan}
\textbf{Purpose:} The purpose of this test is to detect errors and test the functionality of the system.

\textbf{Problem statement:} The problem that will be tested will be if the participant can register a new user, alongside if the participant can log in with the user he/she created. Another thing that will be tested is if the participant can navigat back to the home page/item page. More details of the exact tasks are found in "Task list" bellow. 

\textbf{Test roles:} In this test it will be two different roles. One moderator and one observer. The moderator will be the one who heads the test. With that it means the moderator will read out the problems/task that the participant will be doing and also give out follow-up questions. The moderator role will be semi-open. That means the moderator will not help the participant with the test, but can can answer questions that the participant have, if there are some unclear things about the test task.

The observer role will be completely passive. The only thing the observer will do is to note down some data that is gonna be collected by this test. It can be a certain behavior that the participant have through the test, like wrong "click" or how long it takes for the participant to complete the test. Other behavior which can affect the test result will also be noted by the observer. Since it will be used a Thinking Aloud method, the observer will need to note down what the participant is thinking when doing the test. 

\textbf{User profile:} The user that will be tested is two persons that work at MakerSpace and the group will also find one other student on the school for example, a teacher student.

\textbf{Methodology:} 
The group has decided that they will divide the test in this four steps:
\begin{enumerate}
    \item \textbf{Background questionnaire:} This is where the moderator for the test collect background information about the participant before starting the test. This is going to be be gender, age and what education they take, or whether it is a student or a teacher. The moderator also make sure that the anonymity agreement will be signed before starting \textit{(see appendix "Use test template")}.
    
    \item \textbf{Orientation:} The moderator will read and give out different problems/tasks that the participant will perform.
    
    \item \textbf{Performance test:} When the participant is preforming the test, then the observer will take notes of different things that has been agreed to collect.
    
    \item \textbf{Participant debriefing:} After the test is done, the moderator will ask some follow-up questions. This is to find out how the participant felt the test went or if they have anything else they want to add before finishing the test.
\end{enumerate}
\textbf{Task list:}
\begin{table}[h]


\begin{tabular}{lllll}
 \textbf{Nr.} & \textbf{Task} & \textbf{Claims} & \textbf{
Measure}  \\
\textbf{1.} & Create a new user &  Use drop down menu & Time, errors \\
\textbf{2.} & Log in with user & Use drop down menu & Time, errors  \\
\textbf{3.} & Get back to homepage & Use drop down menu & Time, errors 
\end{tabular}
\end{table}

\begin{table}[]
\begin{tabular}{ll}
\textbf{Nr.} & \textbf{Description} \\
\textbf{1.} & You want to make a new user on MakerSpace Management System, how will you do it? \\
\textbf{2.} & You will now log in with the user you created, how will you do it? \\
\textbf{3.} & You are now logged in and will look at equipment that MakerSpace have, how will you do it? 
\end{tabular}
\end{table} 

\textbf{Test environment and equipment requirements:} The test will be carried out in a group room at the school. That will make sure that the participant will not be disturbed while conducting the test, as that the group room is a quiet place.  

Equipment that is needed is a computer with the latest functioning version of the MakerSpace Management System. A few pens is needed and also a physical copy of the test tasks, one to the moderator and one for each of the observers and participant. A paper where the moderator collect background information, and a something the observer can takes notes on.

\textbf{Evaluating measure:} The main thing that will be tested or be collect is if the system work. The group will check if the participant is able to register a new user and log in with it. The observer will take notes of all the errors the participant do. For example if the participant can't find the log in site or don't understand how or what he/she is going to fill out. The time it takes for the participant to complete each task will also be measured.

\subsection{Summary}
The age of the participant was between 20-30 years. Two of the participant is studying IT and one studying teacher. For all of the participant the whole test went very fast. For each task all of them spent between one minute and 30 seconds to about 40 seconds to complete the task. 

On task one there was none of the participant that successfully could create a valid user. That was something that was expected, since the group had not added use feedback. Otherwise did none of the participant have any issue to find and fill out the information. Although none had any big issue finding registrar, they still think it was a bit silly to have the link in the drop down, instead of a menu link. The group will consider taking out both log in and registrar from drop down to make it even easier for the user to find it.

Task two went well for all of the participant, after the monitor help the participant to make a valid user before moving on to this task. None had any big issues, only as in registrar that they wish it was in the header menu instead of drop down. Another feedback was that one of the participant wish the system had a enter function, so it is possible to just hit enter when done filling out instead of clicking the actually log in button.

The last task went well too, but although all of the participant found back tho the right page, they was facing some confusion. This was also something that was expected since the was no real item displayed on that page jet. That made some participant to start clicking on other stuff on the page, as available button and so on. Some feedback was that someone tried to click on the logo, cause they thought that would take them back to front page, something it did not. 

On the question of how hard the participant think it was to complete each task, they think the first task was a bit tricky and confusing, but after getting a valid user they had no problem completing the two other tasks. Other feedback was as mentioned more visible menu and enter functionality. The group will take all the feedback into and make the system better. 


\chapter{Discussion}

\section{Work methods - Magnus}
Throughout the project there has been many methods put to work. And all of them has served a purpose for the project driving it forward. The methods put to work for the project have been a project management method, qualitative and quantitative methods, a questionnaire, interview method and a method for testing and evaluating systems. Of all the methods used the project group will discuss what worked and didn't work about them and why.          
\subsection{Hva fungerte?}
Agile as a project management method for the project has worked fine. It has allowed the project group to be flexible with both the development of MMS and writing the report. Being able to go back and forth between parts of the project without worrying if something needed changing or editing that it would put a damper on the progress of the project. Agile suited the project best, especially the focus on the people doing the work and the flexibility between meetings and work and being able to adapt to changes. This has enabled the project group to achieve very good teamwork and understanding of the group members. The project group paired Scrum as a Agile framework and it has worked very well, especially because of the focus on software development and being targeted for teams of three to nine members. The sprints performed in the Scrum framework suited the work method and ethics of the project group and project very well. And if something had to be re-planned or changed, short meetings was easy to arrange since the whole project group was closely located and often at the school.

The Qualitative and Quantitative methods for gathering information has been followed in the places where they have been used. Gathering information for analysis of different technologies and general research is essential to any paper and Qualitative method has been followed without any issues. For the quantitative method used in gathering information through a questionnaire, interview and usability testing went without much issue. A questionnaire was prepared and sent out, and the project group handled the responses as anonymous data to compare and analyze. The project group conducted one interview and the interview method used was a semi-structured interview with a set of prepared questions where more questions could be asked and used to clarify any misunderstandings. It proved to be a learning rich experience and gained a lot useful information by also holding the interview locally at MakerSpace Inspiria. 
The usability testing fell under the heuristic evaluation used to test the project groups own system to find any advantages or disadvantages compared to other similar systems. The heuristic evaluation went without trouble and followed a set of rules that was followed. The usability test that was done on a set of users was done well and the project group had a thorough test plan that was followed and made the testing go smoothly. The test subjects managed to perform the tasks asked of them and followed another method called the "think aloud method", where every action that was made was communicated orally to the one responsible for conducting the testing. The methods used in the project has worked and gone well, but not everything is perfect in any given situation.               
\subsection{Hva fungerte ikke? Hvorfor fungerte det ikke?}
When using Agile or any other form of project management method. The team working on the project must follow the structure otherwise it won't serve it's function. The project group has done well in following Agile while using Scrum. But with fixed meetings with guidance teachers and assignment provider the "scrums" and short meetings haven't been used in a way it should. The project group also in the beginning of the project found weekdays where it was possible to meet and work together. Negating any need for short meetings, as any planning could be done at anytime. Agile software development values working software over comprehensive documentation. This contradicts what is done in the project where the project is judged more based on documentation than the software and the documentation have been prioritized at certain points over software development. Agile was however what was most suited for the project and except from the mentions above served it's purpose. 

The questionnaire that was sent out got twenty responses, and the data collected was valuable and the project got a lot of usable information from it. But the structure and the order of the questions could have been better planned. Some of the recipients answered questions seemingly thinking there wouldn't be a follow up question, and had to either repeat themselves or ignored the question. Splitting the questionnaire into two questionnaires could have been an alternative, where one was for those with a digital system already in place and one for those without. This might a have resulted in more accurate and consistent answers. Although the project group got usable data the questionnaire could have been different and didn't work as well as anticipated. (?) %vet ikke om denne delen trengs eller skal flyttes, spørreskjemaet funket forsåvidt

agile, brukt men ikke helt 

questionnaire svar men kunne vart mer strukturert etc

interview, stregnerre struyktur mer kontroll, flere

\section{Delivery}
For this project the group did set up some main and sub goals for what is going to be delivered in this project. The main goal was to make a inventory module which provides an overview of the equipment available at MakerSpace. The sub goal was to to build a lending module from what the group already had made. Alongside with the main and sub goals, it was made some smaller goals that was made so the group could reach the main and sub goals. In this section it can be read what did the group deliver, and what did not.

\subsection{What has not been achieved?}
Early on the group had the sub goal about lending included in the planning, and it was also made some database model of how a lending system would work, and it was also some discussions about what should be included in that kind of an system. Unfortunately the group started to realize when it was some trouble to get the inventory module to work as the group wanted to, that the time for building a lending module was running out. The group soon started to realize how big task it was to build a lending system is and how complex it is to develop. As that the group understand that situation, it was made a decision that the main focus was going to make a complete innovatory module as possible.

As the end of the project the group can now see that the complete inventory module that was supposed to be made, also did not go as planned. After having some big trouble getting the frontend to work as wished, it delayed the group from reaching the main goal. The group underestimated how complex it is to develop a completely complete innovatory system. After that the group has been looking at different 
solution for how to display the frontend, it was made one solution. The one solution that was made, turned out not work as it was supposed to. Because of that, the group needed to redevelop the frontend 
solution. By that a lot of time has been spent trying to display a working frontend. As the group was having a weekly meetings with both the supervisor and assignment provider, they was both aware of situation and the progression of the project. 

\subsection{What has been achieved?}
Even as the main and sub goals did not go as planned, it was still something that the group could achieve. The group have develop a solid backend with a working application programming interface (API). It has also been set up a working database structure for the inventory system. The group has also 
formulated a questionnaire that has been sent out to other Maker spaces in Norway and have been receiving good result. With the result from the questionnaire it has been collected good idea's that can be use full for further develop.

\section{Hva har vi lært? - Andreas}
- Hva har vi fått ut av prosjektet
- hvor mye jobb det faktisk er
- Hvor viktig det er å bruke 

\section{Hva gikk bra/ikke bra? - Morten??}
\subsection{Hva gikk bra?}
The group believes several tasks were completed well.

Routing ...

An evaluation of Finn.no was conducted.
The group used the knowledge gathered during the evaluation to design and shape the wireframes.
The group believes the wireframes proved useful as each individual group member knew what the final product was to emulate.
The group could thus more easily assign tasks to individual group members and have that group member know what is the expected outcome of the assignment.
The group thus believes the evaluation along with the later designed wireframes proved very useful during the development of the application. 

Singel page app(?) ...

The backend primarily supports two features; fetching items as well as authenticating users.
These two features are exposed through endpoints accessible to the frontend allowing the frontend to communicate with the backend.
The two features are currently not used together - meaning there are no items which require user authentication to fetch.
A desired quality of the application has been since the start of the project to make the application extendable.
These two features shape the core of the application and future development can thus proceed in whatever direction is deemed desirable by the future developers of the project.
The group thus believes the development of the backend went well.

Tester ...

Intervju ...

\subsection{Hva gikk ikke bra?}
Gikk ikke bra
Frontend - Vue

\subsection{Hva ville vi gjort annerledes? }
- Hatt en design student
-flere intervju(magnus added)

\todo{Mål nådd? Fungerte metodene dere beskrev?}
The project has been a journey for the group. With both ups and downs, but it has gone well despite not achieving everything the group had hoped to achieve. In chapter five the project group will discuss and voice opinions on how the project went. This includes what the group think went well, not so well, deviations, general thoughts and what the project group has learned from the project.   

\section{What went not so ducky}
Something that proved a challenge was Vue.js. Vue.js was not that easy as it was thought to be. The single page application feature of Vue.js is certainly a good feature, but the specific code and template that needed to be used proved to be more difficult to handle than anticipated and it worked in the end, but might have been easier using React instead or something that might have been more familiar to the project group to use. CSS was  also a bit problematic as it is something that no one in the group had much experience with. And was one of the more difficult things to get working. Styling and positioning combined with usability proved time consuming and one fixed error resulted in another error. Where it seemed like an endless circle of chasing small errors. But this isn't unusual for software development however it was not planned to have any problems with it. 

cookie section her....----??

During the time of writing the report the project group has used an online editor called Overleaf. When using sources it was originally thought to use APA 6th. This did not happen because the packages Overleaf has for APA 6th and the bibliography the project group used, ended with the whole document reformatting when switching to APA 6th. This reformatting ruined the entire setup both chapters, subsections, picture positions, text formatting and headlines changed drastically. Quite a lot of time was spent fixing and trying to make it work, but in the end it was decided to go back to a non APA 6th solution. 

\section{The groups thoughts}

\section{Deviations from the original plan}
For one we planned on including a loaning-module to the MMS, but the scope for the project had to be re-adjusted and the loaning-module was left to be developed by someone else in the future. So the project group could focus on the core features. The core features being the backend and it's database and the frontend with it's web application, registering new users, logging in and communication with the backend. 

CSS...?
\section{What the group would have done differently}
With the knowledge the project group has now. There is a set of things that could have been done differently. One of these things is setting the report up properly. Overleaf is a fantastic tool with many opportunities, but knowing how to properly set it up and doing it at the beginning would have saved the project group from some problems. APA 6th could have been properly setup if the project group knew doing it at a later point instead of adding a package at the start would ruin the report structure.

When the project group choice to use a full JavaScript stack, Vue.js was chosen as the frontend technology to be used. Vue.js single page application proved to be harder to work with than anticipated and if the project group were to change something, it would probably be something differently than Vue.js. It would be something the project group is more familiar with or not develop around a single page application.  

\section{What the group learned}


 
  



Describes the project process - what went well, what did not and what could have been done differently.
The chapter also outlines what the group has learned during the project.

- catogori ting relasjon

- Får lov å kjøre uten TLS

- Turns of service, legges til av oppdragsgiver(juridiske årsaker) 

- avik(ting vi ikke har fått gjort)
1. ikke laget utlånmodel
2. coockies(?)
3. 
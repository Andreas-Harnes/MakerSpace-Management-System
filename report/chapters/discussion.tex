\chapter{Discussion}
The project has been a nice journey for the group. With both ups and downs, but it has gone well despite not achieving everything the group had hoped to achieve. In chapter five the project group will discuss and voice opinions on how the project went. This includes what the group think went well, not so well, deviations, general thoughts and what the project group has learned from the project.   
\section{What went wrong}
Something that proved a challenge was Vue.js. Vue.js was not the lord and savior it was thought to be. The single page application feature of Vue.js is certainly a good feature, but the specific code and template that needed to be used proved to be more difficult to handle than anticipated and it worked in the end, but might have been easier using React instead or something that might have been more familiar to the project group to use. CSS was  also a bit problematic as it is something that no one in the group had much experience with. And was one of the more difficult things to get working. Styling and positioning combined with usability proved time consuming and one fixed error resulted in another error. Where it seemed like an endless circle of chasing small errors. But this isn't unusual for software development however it was not planned to have any problems with it. 

cookie section her....----??

During the time of writing the report the project group has used an online editor called Overleaf. When using sources it was originally thought to use APA 6th. This did not happen because the packages Overleaf has for APA 6th and the bibliography the project group used, ended with the whole document reformatting when switching to APA 6th. This reformatting ruined the entire setup both chapters, subsections, picture positions, text formatting and headlines changed drastically. Quite a lot of time was spent fixing and trying to make it work, but in the end it was decided to go back to a non APA 6th solution.  
\section{The groups thoughts}

\section{Deviations from the original plan}
For one we planned on including a loaning-module to the MMS, but the scope for the project had to be re-adjusted and the loaning-module was left to be developed by someone else in the future. So the project group could focus on the core features. The core features being the backend and it's database and the frontend with it's web application, registering new users, logging in and communication with the backend. 

CSS...?
\section{What the group would have done differently}
With the knowledge the project group has now. There is a set of things that could have been done differently. One of these things is setting the report up properly. Overleaf is a fantastic tool with many opportunities, but knowing how to properly set it up and doing it at the beginning would have saved the project group from some problems. APA 6th could have been properly setup if the project group knew doing it at a later point instead of adding a package at the start would ruin the report structure.

When the project group choice to use a full JavaScript stack, Vue.js was chosen as the frontend technology to be used. Vue.js single page application proved to be harder to work with than anticipated and if the project group were to change something, it would probably be something differently than Vue.js. It would be something the project group is more familiar with or not develop around a single page application.  

\section{What the group learned}


 
  



Describes the project process - what went well, what did not and what could have been done differently.
The chapter also outlines what the group has learned during the project.

- catogori ting relasjon

- Får lov å kjøre uten TLS

- Turns of service, legges til av oppdragsgiver(juridiske årsaker) 

- avik(ting vi ikke har fått gjort)
1. ikke laget utlånmodel
2. coockies(?)
3. 
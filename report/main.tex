\documentclass[12pt]{report}
\usepackage[utf8]{inputenc}
\usepackage{graphicx}
\usepackage{todonotes}
\usepackage{tabu}
% Aligns text to the left – gets rid of “wall” on the right side
\usepackage[document]{ragged2e}
\usepackage{csquotes}
\usepackage{biblatex}
\usepackage{hyperref}
\usepackage[parfill]{parskip}
\usepackage{pdfpages}
% Times new roman
\usepackage{mathptmx}
\usepackage{styles/presentationpage}
\usepackage{float}
% Use more of the page
\usepackage{a4wide}

\def\category{Development}
\def\credits{20}
\def\area{Information Technology}
\def\free{X}
\def\freeafter{(25/05-2019)}
\def\freeafteragreement{X}

\def\customer{HiØ MakerSpace}
\def\tutor{Terje Samuelsen}
\def\department{Department of Computer Science}
\def\projectnr{BO19-G03}
\def\contact{Michael Lundsveen}
\def\abstract{
    I like trains

}

\def\keywordone{IT}
\def\keywordtwo{Programming}
\def\keywordthree{Web Development}
\def\titlepresentationpage{MakerSpace Management System}
\def\authorspresentationpage{Andreas Harnes, Celina Marie Kristiansen, Magnus Klerck and Morten Offerdal Kvigne}


\bibliography{bibliography/bibliography.bib}

\renewcommand{\listfigurename}{Figures}
 
\renewcommand{\listtablename}{Tables}

\title {
    Report \\
    MMS - MakerSpace Management System \\
    BO19-G03
}

\author {
    Andreas Harnes \\
    Celina Marie Kristiansen \\
    Magnus Klerck \\
    Morten Offerdal Kvigne
}

\begin{document}

\begin{titlepage}

\newcommand{\HRule}{\rule{\linewidth}{0.5mm}} % Defines a new command for the horizontal lines, change thickness here

\center % Center everything on the page

\textsc{\LARGE ØSTFOLD UNIVERSITY COLLEGE}\\[0.5cm] % Name of your university/college
\textsc{\Large Faculty of Computer Sciences}\\[1.0cm]
\includegraphics[scale=.1]{figures/makerspace_logo.png}\\[1cm] % Include a department/university logo - this will require the graphicx package

\textsc{\Large Bachelor thesis}\\[0.5cm]
\textsc{\large BO19-G03}\\[0.5cm] % Minor heading such as course title

\HRule \\[0.4cm]
{ \huge \bfseries MakerSpace Management System}\\%[0.4cm] % Title of your document
\HRule \\[1.5cm]
 
\begin{minipage}{0.8\textwidth}
\begin{flushleft} \large
\emph{Authors:}\\
Andreas \textsc{Harnes},\\
Celina Marie \textsc{Kristiansen},\\
Magnus \textsc{Klerck},\\
Morten Offerdal \textsc{Kvigne}\\% Your name
\end{flushleft}

\end{minipage}\\[2cm]

% If you don't want a supervisor, uncomment the two lines below and remove the section above
%\Large \emph{Author:}\\
%John \textsc{Smith}\\[3cm] % Your name

%----------------------------------------------------------------------------------------
%	DATE SECTION
%----------------------------------------------------------------------------------------

{\large \today}\\[2cm] % Date, change the \today to a set date if you want to be precise

\vfill % Fill the rest of the page with whitespace

\end{titlepage}

\makepresentationpage
\addcontentsline{toc}{chapter}{Bachelor thesis}

\pagenumbering{roman}

\chapter*{Preface}
\addcontentsline{toc}{chapter}{Preface}

This is a bachelor thesis written by three computer science students and one information systems student at Østfold University College during spring of 2019.
The basis for the project is the lack of a central repository providing an overview of the items at Østfold University MakerSpace.
The project aims to provide MakerSpace with said repository in addition to providing students and employees alike with the necessary tools required to search the repository effectively and efficiently.

The group would like to thank Terje Samuelsen for the invaluable guidance and insight he has provided the group.
In addition the group would like to thank Michael Anderesen Lundsveen for providing us with the project as well guiding the group with his technical expertise.
The group would also like to thank Ted Magnus Sørlie for aiding the group with server assistance when needed.
Lastly the group would like to thank the participants of the questionnaire whose insight proved valuable in shaping the project.

\chapter*{Summary}
\addcontentsline{toc}{chapter}{Summary}
With this project the group has been researching how to develop a management system and the possibilities for a lending system for HiØ MakerSpace. Since HiØ MakerSpace don't have a digital system the group has tried to make management system with a lending module. By gathering information about other Makerspaces and evaluating web site with a similar setup MMS could have. It has made it easy to make wireframes and also collecting good idea's for development. 

The methods and technologies used to achieve the product and why it was done through gathering data and analyzing different technologies is documented with why the group ended with the technologies used. For the methods used for gathering data and managing the project and why the methods was chosen with the benefits it will have for the project.  

The implementation of the technologies, how it was done and how  it works from frontend to backend with a single page application and how it is connected. 
Results from evaluation and testing and MakerSpace interview and questionnaire. What was achieved by testing and gathering information. Discussion of how the project went, what difficulties the group had, what was done and how well it went. Ending with the conclusion of the project talking about what has been achieved and how the continuation of the project will be.


% The group have shared work where some had responsibility for the frontend developing and some for the backend. Some of the frontend technology works and it has been made a solid backend for this project. The prototype that was made has also conducted a usability test, where good feedback was receiving. With all the feedback that has been gathering, it has been collected some information and good idea's for further developing of the system that the group have been working at. 
\newpage

\tableofcontents

\newpage

\pagenumbering{arabic}

\chapter{Introduction} \todo{introdusksjons tekst}
In chapter one the information given will be about the introduction of the project group, assignment provider and MakerSpace HiØ. Followed by the point and purpose of the problem the project group has been asked to fix. After that a table of deliveries for the project will be introduced and a full section of methods the project group will use for the project, including methods for project management, information gathering and how to handle gathered information and how evaluation and testing of the MMS will be done. In the end of chapter one the report structure will be found. The report structure contains a short description of what information can be found in every chapter of the report.            
\section{Project group}
The bachelor group consists of the following four IT students:
\begin{itemize}
    \item Andreas Harnes. 
    Computer Science student with interests in networking and programming.
    \item Celina Marie Kristiansen.
    Computer Science student with experience in databases and big data.
    She has previously been deployed at Logiq where she used Databricks to process data.
    \item Magnus Klerck.
    Information systems student with background as a computer electrician where he worked with IT maintenance and hardware repair.
    He also has an interest in cybersecurity.
    \item Morten Offerdal Kvigne.
    Computer Science student with interests in programming and data security.
\end{itemize}
Celina M. Kristiansen, Magnus Klerck and Morten Offerdal Kvigne have previously been working on projects together.
The three had effective collaborations in their former projects which is reflected in the results they produced.
Due to the effective collaboration the three of them decided to form this bachelor group.

Magnus and Morten had both previously worked with Andreas Harnes and they both believed him to be knowledgeable and as such he was brought in.

\section{Assignment provider}
Østfold University College is a university college located in the county Østfold in Norway. Østfold University College has 6485 students across all the different faculties \cite{total_stundets_at_Hiof}. Østfold University College has two campuses which are located in Fredrikstad and Halden. The different faculties are spread across the two campuses. Campus Fredrikstad is the location for the Faculty of Engineering, the Faculty of Health and Social Studies, and the Norwegian Theatre Academy while Campus Halden is the location for the Faculty of Computer Sciences, the Faculty of Education and the Faculty of Business, Social Sciences and Foreign Languages \cite{Hiof_fakta}. 

MakerSpace is a workspace at Østfold University College at Campus Halden. 
The workspace acts as a playground for students and employees who like to use technology to create something \cite{what-is-makerspace}.
MakerSpace is operated by Michael Andersen Lundsveen and 4 student assistants but all students and employees at Østfold University College have access to the facility 24 hours, 7 days a week
\cite{what-is-makerspace}.
Østfold University College describes the drive behind MakerSpace as:

\begin{displayquote}
The purpose of MakerSpace is to offer students a physical space and an engaging environment with the aim to learn something new by encouraging curiosity and eagerness \cite{what-is-makerspace}.
\end{displayquote}
In addition MakerSpace also acts as an area for lectures, courses and experiments.
They also hold basic programming courses for children during the summer.

MakerSpace is equipped with various equipment related to several fields.
Some of the equipment MakerSpace offers is 3D printers, which can be used to print out different figures and shapes in plastic from a 3D model. Soldering irons for making circuits and soldering components, computers with VR headsets, lego robots and remote controlled drones as shown in figure \ref{fig:makerspace}. MakerSpace also has different tools like screwdrivers, workbenches, pliers, RJ45 crimpers to mention some.

Lastly MakerSpace allows students and employees at Østfold University College to borrow equipment.
Students and employees have to note down their name, their contact information and the item(s) they are lending in a book used to keep track of all borrowed equipment whenever they are borrowing equipment.

MakerSpace responsible Michael Andersen Lundsveen is the assignment provider for the project. 
Michael is the leader and principal engineer for MakerSpace Halden.
He oversees day to day operations and he ensures all equipment is in good condition.
Additionally he is responsible for the repairing or the purchase of new equipment should equipment be damaged or destroyed.
Michael also arranges courses for the MakerSpace employees allowing them to learn how to utilize new equipment, he arranges summer workshops for children during the summer vacations and he sponsors certain arrangements and events at Halden University College.

\begin{figure}
    \centering
    \includegraphics[width=115mm,scale=1]{figures/makerspace.png}
    \caption{Remote controlled drone}
    \label{fig:makerspace}
\end{figure}

\section{The mission}
The background for the project is that MakerSpace currently does not have an overview of what equipment they have, or if the equipment is available and where the equipment is stored.
This is problematic for the employees at MakerSpace as they have no way of knowing whether the equipment is missing or if the equipment is damaged.
Even if an employee is made aware of missing or damaged equipment the employee currently has no way of notifying the other employees and as such the employee will have to manually contact the other employees in order to notify them which is cumbersome.

This is also problematic for the users of MakerSpace.
The users have to search through MakerSpace in its entirety in order to find a given piece of equipment.
They have no way of knowing whether the piece of equipment is damaged until they find it.
They also have no way of knowing whether MakerSpace has the piece of equipment or if another user is currently borrowing it.

Lastly the book used when borrowing equipment also proves problematic.
Anyone can inspect the contents of the book and as such the book does not respect the privacy of the users.
This is concerning should the book be stolen, because all the names and contact information of the users will be astray.
Finally the book is not protected in the event of a fire and all information would be lost.

In order to solve the issues outlined above the group intends to develop two modules.
The employees and users will interact with the modules using web pages.
The following two modules are intended to be developed:
\begin{itemize}
    \item An inventory module allowing employees and users to view the equipment MakerSpace offers, the equipment's condition, where the equipment is stored and weather the equipment is available.
    The module will also allow employees to add, modify and delete equipment from the module.
    \item A lending module allowing employees and users to borrow equipment.
    When returning equipment the module also allows the employee or the user to report any damage that occurred whilst the equipment was being borrowed.
\end{itemize}

The modules will be extendable.
This allows future features to be incorporated into the modules without difficulties for future developers.

%What MakerSpace wants is a system that gives Michael and student assistants access to an overview of all the equipment and tools available in MakerSpace. This system must be modular so that it can be further developed later, and in addition to this a inventory system. They also want a lending module that makes it easier to register the lending of equipment and tools.

%What the group is planning to develop is a modular inventory system and a lending module by adopting technology that is in use today.

%\todo{TODO Alt under flyttes til metode}
%\textbf{Investigation:} The group have talked about some technology they could potentially use. Some of the technology the group have been thinking about is: Vue.js, Node.js and MongoDB. The group will do a investigation for each of those and find out if they fit the project or not. The group will also try and find other technologies that can be more applicable for the project.

%\textbf{Development:} The group will create a website that will offer a lending system of equipment that exists at MakerSpace and also make sure that the system can be further developed later on.

%In the development of the system, the group have made some bullet-points as an overview for the project. These are:
%\begin{enumerate}
%  \item Make the website that will be used for the system and make sure it follow the criteria for universal design.
%  \item Design a structure for the inventory storage.
%  \item Make a module for lending.
%  \item User test the system to find bugs.
%\end{enumerate}

\section{Purposes and delivery}
\subsection{Purposes}
%There are two reasons for the project.
%The first is to provide an overview of the equipment available at MakerSpace to the employees and users of MakerSpace.
%This is beneficial as the employees and users now no longer need to search through MakerSpace in its entirety to see if MakerSpace offers the equipment they seek.

%The second reason is to digitize the lending system MakerSpace currently has in place.
%This is beneficial because it is privacy respecting as only those with the necessary permissions can view the records of equipment currently being borrowed. 
%Additionally should equipment not be returned MakerSpace will know who is responsible.
%Lastly the records of borrowed equipment can be place on a backup - meaning the records of borrowed equipment are not lost in case of theft or in case of a fire.

\textbf{Main goal:} Make an inventory module which provides an overview of the equipment available at MakerSpace.

\textbf{Sub goal:} Make a lending module which digitizes the current lending system used at MakerSpace.

%The purposes with this project is to digitize the current lending system. This will make it easier to not only see what equipment that is available for lending but also what equipment that needs to be order more of. This is divided into a main goal and some sub-goal's.

%\textbf{Main goal:} Digitize current lending system and write the report.

%\textbf{Sub-goal 1:} Make a inventory structure with a lending module

%\textbf{Sub-goal 2:} Make sure that the website (e.g front-end, UI,UX) follow the criteria for universal design.

\subsection{Delivery}

Below are the deliveries the group will deliver during the project.
\todo{Hvorfor scaler denne ut av dokumnetet?}
\begin{tabu} to 1.2\textwidth { | X[l] | X[c] | }
    \hline
    \textbf{Date} & \textbf{Delivery} \\
    \hline
    January 18 ,2019 & Pilot report \\
    \hline
    March 08, 2019 & First version of main document \\
    \hline
    April 23, 2019 & Second version of main document \\
    \hline
    April 26, 2019 & Product \\
    \hline
    May 16, 2019 & Finished version of main document \\
    \hline
    May 27, 2019 & Project poster \\
    \hline
   June 03, 2019 - June 05, 2019 & Presentation of the project \\
    \hline
\end{tabu}
    
\section{Method} \todo{Masse ord som besrkiver lite i 1.5}
\todo{gjør hele kapittelet konkret}
For the project the group will use a set of methods for how the group will conduct research, gather information and how to handle the collected information, what kind of project management method we will use and why it fits the project. The methods used for testing the application and how to most efficiently develop it will also be documented here. 

\subsection{Waterfall} \todo{Legg til figur}
A waterfall model is a sequential project management method which is a linear process of project management\cite{Waterfall_model}. It consists of several discrete phases where no phase begins before the prior phase is complete. Much like a waterfall filling lower level pools, the phases of the waterfall model has six phases where one flow over to the other. The six phases are: Requirements, Design, Implementation, Verification, Deployment and Maintenance\cite{Waterfall_phases}. The reason the group didn't choose waterfall as a project management method is because of the nature of waterfall. The sequential flow of waterfall makes it "impossible" to go back to previous phases. This can be very challenging when going between software development and writing the report. Revisions and changes in both of these can happen at anytime, especially if the scope of the project is changed or planned parts of the projects change during the course of the project.
\begin{figure}
    \centering
    \includegraphics[width=100mm]{figures/waterfallMethods.png}
    \caption{How waterfall works}
    \label{fig:Waterfall_model}
\end{figure}
The right method needs to be used on the correct project, waterfall is most appropriate to use in some situations where:

The requirements are well documented, clear and fixed\cite{SDLC_Waterfall_Model} and the product definition is stable. If the technology is clearly understood and not dynamic, with no ambiguous requirements the Waterfall method could be a good choice. The Waterfall method is also well put in use in situations with ample resources but with the required expertise to support the product and where the project is short. The waterfall model also has several advantages like departmentalization and control where a schedule can be created with deadlines for each phase of the project. As the development moves from one phase to the other and some of the advantages the waterfall model gives are: 
It is simple and easy to understand and use, it is also easy to manage and each phase has specific deliverables and a review process. The Waterfall method is also very suited for smaller projects where the requirements are well understood\cite{Waterfall_Requirement_Advantage}, and the process and results are well documented. 
As with everything the waterfall method has some disadvantages. It doesn't easily allow for changes in the project and when the software reaches the testing stage, going back and making changes to the software can be very challenging if the changes needed to be made haven't been properly documented or thought about in the conceptualization phase\cite{seguetech_waterfall_vs_agile}. Some other disadvantages are: 
There is no working software developed until late in the project and it is not suitable for projects with high risk of changing its scope or other parts of the project\cite{Waterfalls_clear_advantage}. Another disadvantage especially for the project group is that integration for software is done quickly and at the end of the project, which makes it hard to identify any technological or other challenges early\cite{Waterfall_Software_Engineering}. 

So with the waterfall model there are some clear advantages that fits the MakerSpace Management System project.The project is short and the requirements for the project are well documented, clear and fixed. Waterfall works well for smaller projects like MakerSpace Management System and the sequential and step by step phases would give the project more straight forward and easily arranged tasks to complete, which in turn makes the processes and results well documented. And that is very optimal for writing the report for the project. However the waterfall model creates certain difficulties and has some disadvantages that makes waterfall model not the best suited model compared to Agile for this project\cite{seguetech_waterfall_vs_agile}. 

The waterfall model creates a set of challenges. One of these are the development of software isn't started until late in the project. As a lot of the documentation is produced based on the development of software and not only project specifications and analysis of potential technologies i.e. The project has a risk of changing based on revisions on both the software and the report. As progress on both mirrors each other, if changes must be done to the software, some of the documentation might be re-written. This is especially difficult because of the sequential nature of the phases. Where you cant go back to previous phases after they are complete. The scope of the project could also be changed based on time left on the project or challenges with certain components. 

The Agile model is more lenient with making changes on parts of the project and scope. And going back and forth between phases poses less of a problem. Integration of the software in the waterfall model creates a difficult problem as it is integrated late in the project life cycle and gives the group less time to test and identify technological problems and fix them\cite{Waterfall_Model:_What_is_it}. By starting early by implementing and testing the software and making changes to what works and doesn't. It is easier to iron out problems and make sure the release is as smooth as possible. This is best done with Agile, as it works better at handling constant changes in development without causing further problems and risking the entire project. Based on the type of project and possible difficulties, the waterfall model is not the most suitable model for the MakerSpace Management System project.
%\begin{itemize}
    %\item Requirements are very well documented, clear and fixed.
    %\item Product definition is stable.\cite{SDLC_Waterfall_Model}
    %\item Technology is understood and is not dynamic.\cite{SDLC_Waterfall_Model} 
   % \item There are no ambiguous requirements.\cite{SDLC_Waterfall_Model} 
  %  \item Ample resources with required expertise are available to support the product.\cite{SDLC_Waterfall_Model}
 %   \item The project is short.\cite{SDLC_Waterfall_Model}
%\end{itemize}

%\begin{itemize}
 %   \item Simple and easy to understand and use.\cite{SDLC_Waterfall_Model}
  %  \item Easy to manage due to the rigidity of the model. Each phase has specific deliverables and a review process.\cite{SDLC_Waterfall_Model}
   % \item Phases are processed and completed one at the time.\cite{SDLC_Waterfall_Model}
    %\item Works well for smaller projects where requirements are very well understood.\cite{SDLC_Waterfall_Model}
    %\item Clearly defined stages.\cite{SDLC_Waterfall_Model}
%    \item Well understood milestones.\cite{SDLC_Waterfall_Model}
%    \item Easy to arrange tasks.\cite{SDLC_Waterfall_Model}
%   \item The process and results are well documented.\cite{SDLC_Waterfall_Model}
%\end{itemize}

%\begin{itemize}
%    \item No working software is developed until late in the project life cycle.\cite{SDLC_Waterfall_Model}
%    \item High amounts of risk and uncertainty.\cite{SDLC_Waterfall_Model}
%    \item Poor model for long and ongoing projects.\cite{SDLC_Waterfall_Model}
%    \item Not suitable for projects with high risk changing.\cite{SDLC_Waterfall_Model}
%    \item Hard to make room for changing requirements.\cite{SDLC_Waterfall_Model}
%    \item Adjusting the scope of the project can endanger and possibly end the project.
%    \item Integration of software is done very quickly at the end of the project and makes it hard to identify any technological or other challenges early.
%\end{itemize}

\subsection{Agile} 
After looking at the Waterfall model and the Agile model the group has decided to use an agile model for the software development process. After a discussion with the assignment provider about a suitable project management method. The group proposed Agile as the most suitable for the project and the most effective for the type of software development the group would do\cite{Benefits_of_Agile}. The assignment provider agreed to the use of Agile.  
The project is small in scale and is not tightly integrated with any other system. This means the comprehensive documentation a waterfall model requires is not needed before software development can start. Even though Waterfall can be well used for small projects. Additionally this means informal communications between group members work well as the group is co-located.
The agile model does however require a higher presence of the assignment provider as he plays an integral part in which components the group are to prioritize\cite{high_quality_agile}. \todo{Kokludere til slutt}

The assignment provider is involved and accessible throughout the project development.
This allows for incremental development of the requirements set out by the assignment provider which means contract negotiations are not necessary.
Consequently the group can more easily adapt to changes in requirements as the documentation remains largely unaffected by the respective changes.
This is the driving factor behind the agile model:
\begin{displayquote}
Agile is the ability to create and respond to change.
It is a way of dealing with, and ultimately succeeding in, an uncertain and turbulent environment \cite{what-is-agile}.
\end{displayquote}
This is also one of the factors which separates the agile model approach to that of other software development models:

\begin{displayquote}
One thing that separates Agile from other approaches to software development is the focus on the people doing the work and how they work together.
Solutions evolve through collaboration between self-organizing cross-functional teams utilizing the appropriate practices for their context \cite{what-is-agile-software-development}.
\end{displayquote}
With everything taken into consideration the group believes an agile model is more suited for this project than that of a waterfall model.

The group believes the best way to organize the project is using the agile framework Scrum.
Scrum is a framework that is best suited for teams with a size of seven or less \cite{software-engineering-scrum-size} which makes the framework ideal for the group as the group consists of four members.
The assignment provider has set forth the features he would like to see in the system and these features make up the product backlog.
Given the fact that the assignment provider is so involved and accessible it seems fitting that he is involved in deciding which functionality is to be prioritized which Scrum allows him to do as a product owner.
The group believes the flexible approach an agile model used through Scrum provides the best foundation for the successful completion of the project.

%SCRUM (source: https://www.mountaingoatsoftware.com/agile/scrum) will be used as a framework to manage the project and the software development. SCRUM is a derivative of the Agile methodology, but we will use Agile (source:https://www.agilealliance.org) as the methodology when developing software for the project. The Agile method will be used so the group can adjust and adapt to changes in the development as the project goes along. By scaling this to the scope of the group and the project, Agile will fit the development of the project well. 


\subsection{Qualitative and Quantitative methods} When working on the report and documentation for the project it will mainly follow qualitative research methods for gathering information and comparing/analyzing data. Qualitative research method is a scientific method of observation to gather non-numerical data. The aim of a qualitative research project may vary with the disciplinary background, like a mechanical engineer seeking in-depth understanding of a combustion engine for example\cite{Qualitative-Research}. Quantitative methods is when you objectively measure the statistical, mathematical, or numerical analysis of data collected through polls, questionnaires, and surveys, or by manipulating pre-existing statistical data using computational techniques\cite{Quantitative_Methods}.   
 
 Qualitative methods are best for researching many of the why and how questions of something \cite{Qualitative-Research}. Qualitative research is widely used by political science, social work, and education researchers and only produce an explanation for the particular case that is studied. All the qualitative research done for this project will be focused on how we will do something and why that is the best way to do it. Like why did the group choose a certain program over another and how we are going to use said program. The reason for this is that the project does not need any quantitative data for the documentation of the project, two exceptions that might occur is in the testing phase and in a questionnaire that will be sent out. The information gathered from the questionnaire will be about different MakerSpaces in Norway. In the testing phase the project group will ask for a number of students and/or employees to test the developed system and it's ease of use i.g: UI (User Interface), UE (User Experience) and possible bugs. A User Interface is the space where the user (humans) interact with computers (machines)\cite{user-interface}. And the purpose of a User Interface is to make the interaction as effective as possible for a person. Some UI layers could be tactile (touch), visual (sight) and auditory (sound). These layers usually makes up what we know as Graphical User Interface (GUI)\cite{GUI}. The User Experience is about how good, easy and accessible it is to use the competed product. This has everything to do with the customer, and the end user interacting with the product. In the UX the interface, graphics and design is taken into account and planned to be as best as possible\cite{UX}.  
 The data collected from testing could be used to statistically to show that the UI and UX is good for all potential users. No matter how experienced the users are with IT or how many times the users have used the system.

 \subsection{Questionnaire}
 A questionnaire will be made with a selection of different questions aimed at different MakerSpace's in Norway. The questions will be about how they run their MakerSpace and what kind of systems they use, digitally or other to handle their day to day activities. They will also get questions about the equipment they have, what kind of equipment and how they store said equipment. Whether they let people borrow things from them or not and if they have a digital system or any other means to handle equipment being lent out. 
 The questionnaire will be sent out through Google Form to as many Norwegian MakerSpace emails we can find. This is also how the information will be collected. The mail being sent out will kindly ask if they are interested in answering a questionnaire about their MakerSpace. A link will be included to the Google Form. 
 When the answers for the questionnaire comes in. One or more persons from the Bachelor group will go through all of the answers manually. The answers will be compared to each other to map out the differences and similarities between the MakerSpace's. After the answers have been read and compared to each other, they will all be analyzed to see if we can benefit from the answers. By either getting ideas for new functionality, seeing the potential of offering the project groups solution to other MakerSpace's that doesn't currently have a digital inventory solution. Or avoiding issues other MakerSpace's have with their digital inventory solution. The questionnaire will be added as an addendum.          
 
 \subsection{Interview method}
 
 Interviews can be defined as a qualitative research technique. This involves conducting intensive individual interviews with a small number of respondents to explore their perspectives on a particular idea, program or situation.\cite{Interview-Methods} Interviews have three different formats: structured, semi-structured and unstructured.

Unstructured interviews is conducted without any questions prepared before the interview takes place and data is collected in an informal way. This makes unstructured interviews very unreliable from a research standpoint. Answers given during an interview will also vary based on the formulation of the question. Comparing the answers for the questions will then be a challenge due to the different formulation.

Semi-structured interview is like a mix of structured and unstructured interviews. Questions are prepared before an interview and additional questions might be asked during the interview to either clarify, specify or expand on a certain question. 

Structured interviews consists of a set of predetermined questions that all of the interviewees will answer. The questions will be asked in the same order for every interviewee. The data collected from the answers will make data analysis easier because comparing the questions is straightforward. And it is easier to see the contrast between the different answers given.

Interviewing people for the project will have many advantages like the chance to collect detailed information about what other MakerSpaces do, and what kind of systems is in place for loaning out equipment or if they even have a system. By interviewing different MakerSpaces the group has more direct control over the flow of the interview process and has a chance to clarify issues during the interview process. This will make the data collected more accurate and easier to compare to the answers of other MakerSpace branches.

There is however some disadvantages of conducting interviews. One of the disadvantages is the time needed to arrange the interviews at appropriate times for the interviewees and the project group. Each member of the project group would learn most by being present in all the interviews. Being able to see and listen to what other MakerSpaces do would be a better experience than reading a summary or written answers to the interview. Depending on the time needed for other tasks for the project and the interview. The project group will send one person to all the interviews if time is limited.
 
\subsection{Heuristic evaluation}
In a process of developing a system, it can be necessary to do some kind of evaluation and testing\cite{heuristic-evaluation} (see also section below) to get the best end result for a system. One method that the group find suitable for this project is heuristic evaluation. With a heuristic evaluation the group can look into other makerspace system or other systems that have the same similarity and find advantages and disadvantages with how the systems has been designed.

The main point of an heuristic evaluation is to get an expert overview of a system and try to find mistakes that's been made 
under development of the system, this in order to either help improving that system or avoid doing the same mistakes when developing a new system. Alongside trying to find mistakes, you can also with an heuristics evaluation use it to find things that's been done good with the system, and transferring the good ideas to the new system that is under development. 

When conducting an heuristic evaluation, you need some kind of usability principles that is called "heuristics"\cite{heuristic-evaluation}. You can make your own heuristic principles to find out what kind of usability and design you want to look into, and what will be important to look at when you are designing and developing a system. When you are done with the evaluation, you will have a better overview of the system that was been looked at, and what elements that was good and what was bad. By that as previously mentioned you can see what can be improved or what ideas that can be good to have when developing something new. An heuristic evaluation can be a good foundation for developing wireframes but it is important to now that this kind of method doesn't cover everything, so it can be good to combine this method with other tests and methods.\cite{heuristic-evaluation} A good example can be to combine a heuristic evaluation with a usability test method, that can be read about in section below.

If it is impossible the group will try to find other makerspace systems or other systems that have the same similarity, that can be evaluated through an heuristic evaluation, to find out what kind of mistakes or good elements that other systems have, so the group can get some good ideas that they might avoid or implement in the system that is being developed. If the group don't find any other makerspace or other systems that has this kind of a digital system, the group will if possible do an evaluation on a previous bachelor thesis who tried doing the same project as this one. By looking into a previous attempt, the group will see what went wrong and what the other group did well.\cite{heuristic-evaluation}

\subsection{Usability testing}
When the group is done with making a prototype for the system, it can be smart to run a usability test on it \cite{usability-testing}. The main purpose of this kind of testing is to predict the expected performance of an customer using the current system and to detect serious errors before release of the system.\cite{usability-testing}
It will help find difficulties with the prototype and give the group the opportunity to correct the errors before delivery.   

The usability test will need to consist of a test plan (read about the test plan in attachment) that will be followed through the testing \cite{usability-testing}. The test plan consist of a recipe off how to 
to plan and complete the testing and what information that will be collected\cite{usability-testing}. 

The test can be run as many times as desired. In the process of developing a system it can help to run another usability test after fixing the errors from the first test, to detect any new errors that may have appeared after developing some more.

\subsection{Think Aloud method}
Think Aloud is a testing tool that will be used when doing the usability test. \cite{Think_Aloud}
How the tool is supposed to be used is by letting the participant to think aloud when doing the test task he or she has been given. By letting the participant to think aloud will make it easier to discover what the participant really think about the design and the system that has been made. \cite{Think_Aloud}
It will then be easier to know what elements that need to be change to best fit the users needs. 

Some good benefits that the Think Aloud method will offer is: \cite{Think_Aloud} 
\begin{itemize}
  \item \textbf{Cheap.} This method do not need any special equipment to be able to use it. The only thing that will be needed is someone that can sit next to the participant and take notes from what he or she thinks. 
  \item \textbf{Flexible.} It is possible to use this method wherever in a process of developing. Either if it is just a paper prototype or a fully implemented running system. This method can also be used on everyone that is able to tell what they are thinking when doing the test, so that makes it possible to also use this method with users who are visually impaired.
  \item \textbf{Convincing.} When given a direct respond from users and what they think about the design, will be for someone more convincing on what has been tested need a change.
  \item \textbf{Easy to learn.} There are not that many steps to learn, before someone can use this method when doing a usability test. 
\end{itemize}

Although there are many good benefits with the Think Aloud method, it does not mean that there are no downsides. Here is some downsides by using the method: \cite{Think_Aloud} 
\begin{itemize}
  \item \textbf{Unnatural situation.} For most people, they are not used to go around and talk to them self all day. This might makes it hard for the participants to keep up the thinking aloud when doing the test. 
  \item \textbf{Filtered statements.} For some people they want to appear smart when doing the test. This make them to think more detailed before answering, when it is preferred to be given the answer direct. Because of that it is possible to get different result then expected or needed. 
  \item \textbf{Biasing user behavior.} Very often can interruptions change user behavior. In such cases, the behavior doesn't represent the real use, so one cannot base the design on the outcome.
  \item \textbf{Not enough.} Although this Think Aloud tool brings many good benefits, it is not enough to use this tool alone. It is preferable to combine several different tools to get the whole picture of testing.
\end{itemize}

\newpage
\section{Report structure} 
\textbf{Chapter 1, Introduction:}
% Chapter 1.1 contains information about each member of the project group to give more information about who is working on the project. Chapter 1.2 gives into about the assigment provider in this project; Østfold University college. This chapter also contains information about our contact at Østfold University College, Michael Andersen Lundsveen. Chapter 1.3 describes why the project was created, wishes for the project and a scope over the project. Chapter 1.4 is divided  into two sub chapters. The first sub chapter is 1.4.1 Purpose. In 1.4.1 The Goals of the project is described. In chapter 1.4.2 Delivery all the important delivery dates are listed. Chapter 1.5 Method describes the different work methods and approaches that was researched and selected for this project. Chapter 1.6 is the Structure of the report. In this chapter the order and content of every chapter in the report is explained. 
It contains information about the members of the project group and the assignment provider. This chapter also explains why the project was created, wishes for the project and a scope over the project. Here you will also find all the goals and the important delivery dates. Lastly this chapter describes the different research and working methods and approaches that was researched and selected for this project and the structure for the rest of the report. 


\textbf{Chapter 2, Analysis:}
% Chapter 2.1 contains information about different MakerSpace's, what systems they use and how the lab is organized. Chapter 2.2 contains information about all the database technology the project group chose and why it is being used. It also contains information of other database technology that is not being used and why. Chapter 2.3 contains all information about the Frontend technology selection. What the project group chose and why and also other alternatives that were not suitable for the project. Chapter 2.4 contains information about style sheets that are being used, what kind of CSS and why it is being used. Chapter 2.5 contains information about the Backend technology, what kind of technology is being used and why. Chapter 2.6 contains information about how the Frontend and Backend is connected. This includes HTTP and HTTPS, request methods and status codes. Chapter 2.7 contains information about Git and Github, what it is and how it is being used.   
It contains the analysis part of the report. The chapter includes the analysis of the systems used in other Makerspaces and how their equipment is organized, the analysis of different database systems and the groups choices for the project. Description of the analysis for the frontend- and backend architecture and the selection of the technology chosen. And information about other technologies and protocols used in the project.

\textbf{Chapter 3, Design and implementation:}
% Chapter 3.1 contains information about the Frontend and it's design and how it is implemented to look and work the way it does. This includes design choices for both the look and functionality of the web application. Chapter 3.2 Backend contains information about the design and functionality of the Backend for the end-user and what it provides them, with a good look at Endpoints and how it functions. Chapter 3.3 contains information about Frontend and Backend connection and how communication between each other is designed and function. Chapter 3.4 Database contains information about the implementation and design of the database and how it functions.    
The chapter contains information about the design choices and the functionality of the frontend- and the backend architecture are explained. This chapter also contains information and documentation about how the frontend and backend are communicating and the implementation and documentation of the database system.

\textbf{Chapter 4, Testing and Evaluation:}
% Chapter 4.1 contains information about the Heuristic evaluation method to evaluate the project groups web application by comparing it to a similar web application. Chapter 4.2 contains information about the testing of the web application and it's functionality by asking a group of people to use the web application and a detailed test plan.    
It contains information about the different testing methods and interview methods used during testing the application and the result of the testing.

\textbf{Chapter 5, Discussion:}
% Chapter 5.1 contains information about the project groups thoughts on what went well with the project. Chapter 5.2 contains information about the project groups thoughts on what did not go as well as they thought it would and complications during the project. 5.3 contains information about what could have been done differently during the project and what lead the project group to think what they decided to do, could have been done another way. Chapter 5.4 contains information about deviations from the original plan and idea the project group had before and during the project. What was included and what had to be dropped or couldn't be done in time is also part of this chapter. Chapter 5.5 contains information about what the project group learned during the project. What will they remember best and what will they take with them further into work life.
It contains information about the project groups thoughts from the project, complications and what went well in the project. In this chapter you also find information about what the group would have done differently and any deviation from the original plan. In the last part of this chapter there is information about what the group learned during the project.

\textbf{Chapter 6, Conclusion:}
% Chapter 6.1 contains information about the results of the project. What did the project group deliver, does the web application cover the necessary functions and is the developed application usable. Chapter 6.2 contains information about the continuation of the project. What can be built upon the already existing web application and how to hand over access to MakerSpace HiØ. 
It contains information about the results of the project. In this chapter there will be information about the final application, the state of the web application and information about continuation of the project of the application after this project is done.


\chapter{Analysis}
\todo{Skrevet om, trengs å leses over}
This chapter provides the background knowledge required to understand the thought process behind the selected technologies.
The group has been researching different technologies and solutions that can be viable for the application we are developing. Throughout the selection process the group have presented the assignment provider, Michael A. Lundsveen with results from the research and recommendations the group has found for the different technologies that can be used for the application. The group together with Michael worked together to select which technologies that should be used based on the this research.

The chapter first inspects the database technologies that are suitable for the assignment.
Afterwards the chapter examines the selected frontend technology before analyzing the technology stack used for the backend development.
Lastly the chapter provides insight as to why the methodology selected is the most suitable for the project.

    

%Ut ifra spørreskjema så skriver vi hva vi fikk ut av det og hva vi lærte av det. Vi har ikke sendt ut spørreskjema enda så denne delen er av den grunn ikke fylt ut.

\todo{Why only looking at the two most popular? Why not others?}
\section{Database technology selection}
The two most popular database systems today are the SQL- and NoSQL-based database systems \cite{stackoverflow-db-statistics}.
The distinction between SQL and NoSQL database systems have become increasingly blurred \cite{sql-vs-nosql}, but there are still key differences between the two which makes them worth analyzing in order to find the most suitable database system for this project.

\iffalse
\subsection{CAP theorem}
The CAP theorem is a theorem within distributed database systems \cite{sql-schema}.
The theorem states that only two out of the following conditions can hold at any time:
\begin{itemize}
    \item Availability.
    This condition states that every request receives a (non-error) response.
    This means the data requested may not necessarily be up-to-date as the node may not have the up-to-date data \cite{sql-schema}.
    \item Consistency.
    This condition states that all nodes have access to the same data.
    This means whenever an attempt to extract data from the database is made the database will either provide up-to-date data or failure \cite{sql-schema}.
    \item Partitioning tolerance.
    This condition states that the database system will continue to operate despite any number of messages between the nodes being delayed or dropped by the network.
    This means the database system can operate normally while sustaining any number of network failures so long as not every node is experiencing failure \cite{sql-schema}.
\end{itemize}

A database is said to support AC when availability and consistency are selected, its said to support AP when availability and partitioning tolerance are selected and its said to support CP when consistency and partitioning tolerance are selected.
\fi

\subsection{SQL} 
SQL(Structured Query Language) is a database language for data management\cite{sql-goal} of a RDBMS(relational database management system) \cite{sql-is-a-rdbms} based on the relational data model proposed by Edgar Frank Todd \cite{rdbms}.
The relational data model logically structures all relations(tables) and each relation is provided a name and is built up of named attributes - columns of data \cite{sql-is-a-rdbms}.
The rows of a table are known as tuples \cite{sql-is-a-rdbms} and, provided the table is normalized for the first normal form or higher, they contain one value per attribute \cite{sql-1nf}.
Figure \ref{fig:relational-db-visualised} illustrates the relational model graphically.

\begin{figure}
    \centering
    \includegraphics[width=115mm,scale=1]{figures/relational-db-visualised.png}
    \caption{Table named "Customer" containing three attributes and four tuples}
    \label{fig:relational-db-visualised}
\end{figure}

SQL offers a data definition language(DDL) \cite{sql-components}.
The DDL defines the database schema - that is the structure, meaning which attributes the tables consists of, and which tables the database consists of.
It is impossible to add data to the database until a schema has been defined.
The DDL also describes any relational integrities the tables of the database, or the columns of table, may have \cite{sql-constraints}.
In figure \ref{fig:relational-db-relation} a given customer may have several orders but a given order may have one and only one customer associated with it; this is an example of a one-to-many relationship.
In addition to a one-to-many relationship an attribute or table may also have a one-to-one or a many-to-many relationship \cite{sql-relationships}.
Lastly the DDL also defines any security constraints the database may have - such as which users have permissions to read, update or delete the contents of a given table \cite{sql-ddl}.

\begin{figure}
    \centering
    \includegraphics[width=115mm,scale=1]{figures/relational-db-relation.png}
    \caption{One-to-many relation between "Customer" and "Order"}
    \label{fig:relational-db-relation}
\end{figure}

SQL also offers a data manipulation language(DML) \cite{sql-components}.
The DML allows for the insertion of, retrieval of, update of, and deletion of data \cite{sql-dml-options}.
Whenever a tuple is added, updated or removed the tuple must conform to the restraints set by the DDL.
The attempted action will be aborted should there be discrepancies between the constraint of the table or column and the data.
This ensures that the data in the table is accurate and reliable \cite{sql-constraints}.

In addition SQL:
\begin{itemize}
    \item Provides an organized structure as data is defined once, and items referencing the data does so using foreign keys \cite{upwork-sql-adv}.
    \item Reduces data redundancy due to the organized structure \cite{upwork-sql-adv}.
    \item Allows JOIN operations \cite{upwork-sql-adv}.
    The operation allows two or more tables to be combined based on some shared attribute between them.
    The JOIN operation provides the ability to fetch specific data across tables that otherwise would prove cumbersome \cite{sql-joins}.
    \item Provides support for transactions.
    Transactions are essential in databases which require reliable data as they provide a framework for an all-or-nothing approach.
    This means the sequence of involved operations have to succeed in its entirety.
    The database will perform a rollback and raise an error should one of the operations fail \cite{sql-transactions}.
\end{itemize}

\subsection{NoSQL}
A NoSQL(Not only SQL) database provides data storage and data retrieval capabilities that is not modeled using a conventional RDBMS \cite{nosql-not-rdbms}.
Schema-less models are used instead of the traditional RDBMSs.
The most popular models include:
\begin{itemize}
    \item Key-value stores.
    Every item in the database is stored as an attribute name with a corresponding value \cite{mongodb-explains-nosql}.
    The key and value can be anything, and the key acts as a unique identifier for the value \cite{nosql-key-value}.
    \item Document databases.
    Document databases are an extension to key-value stores.
    Documents are structures which can contain many different key-pairs, and documents can even contain other documents.
    The documents can store data in different format, such as XML or JSON.
    Document databases are intended to store semi-sorted data \cite{nosql-document-sort}.
    \item Wide-column stores.
    The data in the database is stored in columns rather than the typical SQL rows.
    A given row can therefore have columns that other rows does not have.
    A wide-column store can be considered a two-dimensional key-value store \cite{infoworld-sql-vs-nosql}.
    \item Graph stores.
    The data in the database is represented as a graph.
    Their intended use is to traverse and navigate relationships.
    These databases use nodes to represent items in the database, and an edge between two nodes represents a relationship between the two items \cite{nosql-graph}.
\end{itemize}

NoSQL is advantageous because it:
\begin{itemize}
    \item Allows for semi-structured or unstructured data in the database.
    This flexible design allows the database to handle changes in structure more easily, and also allows the database to scale with ease \cite{mongodb-adv-nosql}.
    \item Allows for data to be inserted without a schema \cite{omkarsoft-adv-nosql}.
    This is beneficial should data be expected to change during the development of the project.
    \item Scales horizontally rather than vertically \cite{mongodb-adv-nosql}.
    Scaling horizontally allows partitions of data to be scatted across several pieces of hardware in order to store the data in the database.
    By contrast, scaling vertically means replacing already-existing hardware with more powerful hardware \cite{technopedia-nosql-scale}.
\end{itemize}

\subsection{Why SQL is the selected database system}
The assignment provider believes a SQL database system is more suited for the project than that of a NoSQL database system.
The equipment provided at MakerSpace is limited in quantity - as such the data to be stored in the database is also small in scale.
Due to the limited data the schema is simple to define, and as the requirements for the project are unlikely to change at a structural level the flexibility a NoSQL schema provides is not necessary.
The database is also likely to be stored in a limited number of locations and as such the distributed scaling capabilities a NoSQL database system offers is not particularly beneficial.
Overall the benefits provided by a NoSQL database system are not well suited for the project nor the scope of it which makes the SQL database system the preferable choice due to its organized structure, its relational properties and its support for JOIN operations.

\subsection{MariaDB as software}
MariaDB is the assignment providers database management system of choice.
MariaDB is one of the most popular\cite{mariadb-foundation-about} open source relational database management systems in the world \cite{mariadb-about}.
It is a fork of the also popular relational database management system MySQL\cite{mysql-about}.
MariaDB is made by the original developers behind MySQL \cite{mariadb-foundation-about} and it has a large list of sponsors providing financial support including Microsoft and IBM \cite{mariadb-sponsors}.
MariaDB is based on the SQL database system \cite{mariadb-about-searchdatamanagement}.

MariaDB has several notable features.
It is able to handle small and large data alike making it highly scalable, and it allows for rapid access to the data.
MariaDB releases stable releases, and each new release brings speed and stability improvements in addition to new features.
Additionally it places a high value of security - all data is encrypted by default and whenever critical security issues arise a new release is prepared \cite{mariadb-about}.
Lastly MariaDB offers a Node.js connector allowing applications developed on Node.js to connect to MariaDB \cite{mariadb-node-connector}.

\todo{Ikkje bruk several - ver konkret.}
\section{Frontend technology selection} 

The assignment provider wanted a full JavaScript technology stack and as such there were several technologies that could have been chosen for the frontend development. There are several JavaScript frontend frameworks and libraries freely available and as such this was not a big limitation \cite{JavaScript_libraries}.

\subsection{Vue.js}
Vue.js is a progressive JavaScript framework created by Evan You. Being a progressive framework means the user is able to select what features to include, instead of selecting which features not to include. By having a progressive framework you can make a web application only include the minimum of necessary components to work. This is one of the major features of the framework Vue.js. The core library of Vue.js is heavily focused on the view layer only, and is easy to integrate in already existing projects or other libraries \cite{what_is_Vue}. 

Easy Learning Curve
    All you need to know is html and ES5 JavaScript

Vue components

Native Reactivity

Template

Component scoped css

\subsection{React}
JSX



\subsection{Angular}
TypeScript

Learning Curve steep

\subsection{Our solution}
The group decided to recommend using Vue.js because of three major factors; the size of the framework, live updating in the browser and the low learning curve it has. Vue.js is comparably smaller than many of the other big JavaScript frameworks. It is roughly 58.8 kB in size. Since the framework is so small it will help with the loading speed of the web application \cite{vue-size}. 

The second major point for choosing Vue.js was the reactivity of Vue. With the reactivity of Vue.js you could update individual components based off of which data had been updated in the database. In Vue.js each component has getters and setters that enables Vue to track each component for when it's been accessed or modified. Each component also has a watcher instance that will be notified when that components setter is triggered. When this happens the watcher will notify Vue to run the component render function on that component again. With this feature there will be no need to update the whole site to show an update in the database \cite{vue-reactivity}. 

\begin{figure}[h]
    \centering
    \includegraphics[width=115mm,scale=1]{figures/reactivity.png}
    \caption{Reactivity in Vue.js}
    \label{fig:vue_reactivity}
\end{figure}

The last major point for the group was the low learning curve of Vue.js. With this learning curve it means that the easier parts of Vue.js is easier to learn, but the more complicated components can be harder. This is perfect for the groups project since all the group members are new to Vue.js and can get faster into the programming.

The assignment provider agreed with the groups recommendation of Vue.js and allowed to project to be built using Vue.js

\section{Style sheet}
A style sheet is a file that is used in word processing and to define layout style of a document\cite{Style_sheet}.
For Microsoft Word as an example, the style sheets is known as a template\cite{Style_sheet}.
A style sheet contains specifications of a document layout\cite{Style_sheet}. 
The specifications is what page size, margins, fonts and font sizes is going to be set on the document\cite{Style_sheet}.
The group will not dive into what template is going to be used for Microsoft Word, but rather what style sheet is going to be used when developing a web system. Below it can be read what style sheet that will be used in the project and also what units and selectors that will be used. 

\subsection{CSS} \todo{Skriv om SVG og XML + referer til bildet. Gå til Waterfall og referer til bildet der også}
Cascading style sheet or "CSS" is used in any web application one way or another\cite{CSS_Introduction}. %and it will be used in the MakerSpace Management System web application.
CSS provides the visual aesthetics for the webpage by taking HyperText Markup Language (HTML), Scalable Vector Graphics (SVG) or Extensible Markup Language (XML) and converting it into a presentable form\cite{What_is_CSS}. This is done by web browsers applying CSS rules to an HTML document to affect how it is displayed. SVG is a text based open web standard designed to work with CSS and Document Object Model (DOM). It is an XML based text mark up language for two dimensional vector graphics\cite{SVG_definition}. SVG works by defining its data in XML text files and can be searched, indexed, scripted and compressed\cite{SVG_functions}. XML is a text-based format for representing and sharing structured information\cite{What_is_XML}. Figure 2.4 is a simple example of a CSS rule. The "p" stands for "paragraph" and within the brackets we declare that the color(property) is "red"(property value). The result would be the paragraphs text color will be red.
\begin{figure}[h]
    \centering
    \includegraphics[width=80mm]{figures/css-declaration-small.png}
    \caption{CSS rule}
    \label{fig:Css rule example}
\end{figure}
When a webpage is being displayed, what is actually happening is the browser combining the document (HTML) with the CSS information\cite{CSS_and_DOM}. This process happens in two stages, where stage one converts the HTML and CSS into DOM\cite{What_is_DOM}. In figure 2.5 the stages of combining HTML and CSS is shown. The DOM is the PC's memory where the content of the HTML and CSS is combined. Stage two is the browser showing the content of the DOM where the webpage is displayed in all its glory with CSS applied to the HTML.

\begin{figure}[h]
    \centering
    \includegraphics[width=130mm]{figures/DOM.png}
    \caption{HTML and CSS into DOM}
    \label{fig:DOM}
\end{figure}
\clearpage
% bildet plaseres ikke der den skal!!

\subsection{SASS}
SASS or "Syntactically Awesome Style Sheets" is a preprocessor scripting language that is compatible with all versions of CSS \cite{Sass_1}. 
With that it means it is possible to use any available CSS libraries that exists. Sass consists of two different syntaxes, the orginal syntas that they call "the indented syntax" (SASS) and the other one that is a newer syntax called "SCSS" (Sassy CSS)\cite{Sass_2}. 
The first syntax is build like it have indentation to separate code blocks and newline characters to separate rules. The other syntax (SCSS) uses block formatting like in CSS. It uses braces to denote code blocks and semicolons to separate lines within a block. The syntax SASS and SCSS files are also given the extensions .sass and .scss \cite{Sass_2}. 
Below it shows the syntax differences between SASS and SCSS. 

\begin{figure}[h]
    \centering
     \includegraphics[width=115mm,scale=1]{figures/sass_and_scss.png}
    \caption{SASS and SCSS syntax}
    \label{fig:Sass}
\end{figure}


what SASS can offer in which 
CSS alone can't, is that Sass can use variables, mathematical operations, loops, functions, imports, and other  functionalities that make writing CSS much more powerful \cite{Sass_3}. 
This can be good when working with a large and complex sites.

One of the greatest benefits 
by using SASS is as mentioned the ability to use variables. A variable allows to store a value or a set of values, and to reuse these variables through the SASS files as many times as desired \cite{Sass_3}. 
This make the developers job a lot easier as for example if one want to use a color, instead of remember the color code, it can be stored as a describing variable that can be used everywhere in the project.

\subsection{Conclusion}
The project group in addition of agreement from the assignment provider have found out that it will be used SASS for this project. SASS is as mentioned compatible with all versions of CSS, only with more functionality. By choosing SASS the group think it will be taking benefits from its functionality that SASS offers. For this project it will only be developed a simple prototype, but for 
those who will take over the project, it will be nice that they have the opportunity to use this kind of functionality that SASS offers when the project get bigger and more complex. This since the group also have to think about that other can further develop the system.

\subsection{Units}
As the SASS was selected as a style sheet, the group needed also to find out what units to use with the style sheet. Units is what that can be used together with properties. Properties can for an example be length, where it can be used width or margin to set the length of an element \cite{Units_1}. 
Units is what defines the actual length of the element.

It exist two different length units, absolute and relative units \cite{Units_1}.
\begin{enumerate}
  \item \textbf{Absolute units:} The absolute length units are fixed, that means that a length will appear as exactly that size. This can be good when the output medium is known, such as for print layout. On the other hand, absolute length units should not be used on screens, because screens sizes vary so much. 
  
  Some of the units that exists is: \textbf{cm} (centimeters), \textbf{mm} (millimeters) and \textbf{in} (inches).
  \item \textbf{Relative units:} Relative length units is specifying a length that is relative to another length property. Relative length units will scale better between different rendering mediums. In other hand, this is good when working with different screen size, and is much greater to use when one will adjust the SASS to several platforms as mobile, different PC screens and tablets. 
  
  Some of the units that exist is: \textbf{em} (relative to the font-size of the element), \textbf{rem} (relative to font-size of the root element) and \textbf{\%} (relative to the parent element).
\end{enumerate}

As that the group will be working with different screens and platforms and also want it to scale great, it will be relative units that is going to be used for the project. The assignment provider also agrees that this is the best solution.

\subsection{Class and id selectors}
The group will in this project use what called selectors. Selectors are used to find/or select a HTML elements based on their element name as id, class, element name etc \cite{selectors}. 
It will mostly be used id and class names.
\begin{itemize}
  \item \textbf{Id selector:}
The id selector uses a id attribute for a HTML element to select that specific element.
The id of an element need to be unique in 
each single page, with that it means that the id selector is used to select one unique element. To select a element with a id name, it needs to be written a hash (\#) character followed by the id name that has been set \cite{selectors}. 
  \item \textbf{Class selector:}
The class selector works a bit like id selectors. It is supposed to help with choosing a specific element by a name. The differences between id and class selectors, is that class selectors do not need to be unique. It can be several similar names defined in the same files. The good use of this is when there are several elements that need the same changes (for example if there are buttons that needs the same style), it can then be done with defining those elements with same class name, and by setting a specify a style to that class will make all the changes appear for all elements that have that same class name \cite{selectors}.

To select a element with a class name, it needs to be written a period (.) character followed by the name of the class.
\end{itemize}




\section{Backend technology selection}
\todo{en måte: alternativ valg +/- valget}
For the backend technology there are several technologies that could be used. The group had discussed what backend technology to use, and asked the assignment provider to choose. A full JavaScript technology stack was chosen and some of the major backend technologies have been ruled out. Such as Pythons Django, PHP or Javas Spring backend framework. 

JavaScript will be used for the backend, Node.js will be used as the server to create and handle the backend operations needed in the application. Node.js is a JavaScript run-time environment that can run JavaScript code outside of the browser using the V8 JavaScript engine. Node.js runs on a single thread and is able to do this because of the Event Loop. When a request gets sent to Node from the application it will be added to the event queue before being picked up by the Event Loop. The Event Loop will take these requests and send them to the system that Node runs on. Here each request will be sent to different threads and be completed. When the task in done a callback function in the request is activated and the feedback is sent directly back to the application. The reason this event loop is so efficient at handling requests is because it runs asynchronous. So when it receives a request and sends off a callback function for that request. It doesn't need to wait for that callback function to return something before it handles a new request \cite{Node-event-loop}.

\begin{figure}[h]
    \centering
    \includegraphics[width=100mm,scale=1]{figures/event_loop.png}
    \caption{Event loop in Node.js}
    \label{fig:Node_Event_Loop}
\end{figure}

\section{Connecting the frontend and the backend}
The frontend needs to be able to connect to the backend to create, read, update or delete resources.
The connection protocol selected is HTTP(Hypertext Transfer Protocol) and HTTPS(Hypertext Transfer Protocol Secure).

\subsection{HTTP and HTTPS}
HTTP is a protocol designed for communication between distributed systems \cite{mozilla_what_is_http}.
The flow of the protocol looks like this \cite{mozilla_http_flow}:
\begin{enumerate}
    \item The client sends an HTTP request to the target host.
    \item The host receives the HTTP request.
    \item The host processes the HTTP request.
    \item The host sends an HTTP response to the client.
    \item The client receives the HTTP response.
\end{enumerate}

HTTPS is an extension of HTTP \cite{http_vs_https}.
HTTP transmits data in plain text, and as such the protocol is considered insecure as anyone listening or intercepting the traffic can read the data transmitted \cite{http_vs_https}.
HTTPS on the other hand encrypts the data before transmitting it to an external host \cite{https_explained}.
This ensures anyone listening or intercepting the traffic is unable to read the data.
All data can thus pass freely and securely between the two hosts communicating with one another \cite{http_vs_https}.

HTTP/HTTPS lays the foundation for any data exchange \cite{mozilla_http_overview} on the modern web \cite{tutsplus_what_is_http} which makes the protocols the ideal protocols for communication between the frontend and the backend for the project.

\subsection{HTTP request methods}
HTTP request methods indicate the desired action to be performed on the given resource.
There are several HTTP request methods \cite{mozilla_http_status_methods_list} however the application is likely to use some of the following:
\begin{itemize}
    \item GET indicates the client wishes to retrieve the data for the resource identified with the resource identifier provided in the HTTP request, or the client indicates - in the absence of such an identifier - it wishes to retrieve the data for all resources \cite{mozilla_http_status_methods_list}.
    \item POST indicates the client wishes to create a new resource with the provided data in the HTTP request \cite{mozilla_http_status_methods_list}.
    \item PUT indicates the client wishes to update the resource identified with the resource identifier provided in the HTTP request \cite{mozilla_http_status_methods_list}.
    \item DELETE indicates the client wishes to delete the resource identified with the resource indentifier provided in the HTTP request \cite{mozilla_http_status_methods_list}.
\end{itemize}

\subsection{HTTP status codes}
HTTP status codes are responses made by the server in response to a request made by the client indicating either success or failure of the request \cite{http_vs_https}.
All HTTP status codes are separated into the following categories \cite{mozilla_http}:
\begin{itemize}
    \item 1xx represents informational state; The request was received and understood but the server is still processing the request.
    \item 2xx represents successful state; The request was successfully received, understood and fulfilled.
    \item 3xx represents a state of redirection; The client must take further action to complete the initial HTTP request.
    \item 4xx represents client error; The request is invalid in some way and the initial request was therefore not fulfilled.
    \item 5xx represents server error: The server either failed to fulfill a seemingly valid request, refused to carry it out, or is otherwise incapable of carrying it out.
\end{itemize}

There are many HTTP status codes \cite{http_status_codes_list}.
To name a few the application is likely to use:
\begin{itemize}
    \item 200(OK) indicates the request succeeded \cite{mozilla_http_status_codes_success}.
    \item 201(Created) indicates the request succeeded and a new resource was created as a result of it \cite{mozilla_http_status_codes_success}.
    \item 400(Bad Request) indicates the server cannot or will not process the request due to errors believed to be produced by the client \cite{mozilla_http_status_codes_client_error}.
    \item 401(Unauthorized) indicates the client is not authenticated while attempting to perform an action which requires the client to be authenticated \cite{mozilla_http_status_codes_client_error}.
    \item 403(Forbidden) indicates the client is attempting to perform an action it is unauthorized to perform \cite{mozilla_http_status_codes_client_error}.
    \item 404(Not Found) indicates the request requested a resource which does not exist \cite{mozilla_http_status_codes_client_error}.
    \item 409(Conflict) indicates the request could not be processed because the request contained a resource in conflict with another resource \cite{mozilla_http_status_codes_client_error}.
    \item 500(Internal Server Error) indicates the server encountered an unexpected condition which it was unable to resolve \cite{mozilla_http_status_codes_server_error}.
\end{itemize}

\subsection{JSON}

\section{Git and GitHub}

\input{chapters/design-and-implementation}

\chapter{Testing}
The content provided here will be about the testing of the product and the results + how the testing was conducted.

\chapter{Discussion}

\section{Work methods - Magnus}
Throughout the project there has been many methods put to work. And all of them has served a purpose for the project driving it forward. The methods put to work for the project have been a project management method, qualitative and quantitative methods, a questionnaire, interview method and a method for testing and evaluating systems. Of all the methods used the project group will discuss what worked and didn't work about them and why.          
\subsection{Hva fungerte?}
Agile as a project management method for the project has worked fine. It has allowed the project group to be flexible with both the development of MMS and writing the report. Being able to go back and forth between parts of the project without worrying if something needed changing or editing that it would put a damper on the progress of the project. Agile suited the project best, especially the focus on the people doing the work and the flexibility between meetings and work and being able to adapt to changes. This has enabled the project group to achieve very good teamwork and understanding of the group members. The project group paired Scrum as a Agile framework and it has worked very well, especially because of the focus on software development and being targeted for teams of three to nine members. The sprints performed in the Scrum framework suited the work method and ethics of the project group and project very well. And if something had to be re-planned or changed, short meetings was easy to arrange since the whole project group was closely located and often at the school.

The Qualitative and Quantitative methods for gathering information has been followed in the places where they have been used. Gathering information for analysis of different technologies and general research is essential to any paper and Qualitative method has been followed without any issues. For the quantitative method used in gathering information through a questionnaire, interview and usability testing went without much issue. A questionnaire was prepared and sent out, and the project group handled the responses as anonymous data to compare and analyze. The project group conducted one interview and the interview method used was a semi-structured interview with a set of prepared questions where more questions could be asked and used to clarify any misunderstandings. It proved to be a learning rich experience and gained a lot useful information by also holding the interview locally at MakerSpace Inspiria. 
The usability testing fell under the heuristic evaluation used to test the project groups own system to find any advantages or disadvantages compared to other similar systems. The heuristic evaluation went without trouble and followed a set of rules that was followed. The usability test that was done on a set of users was done well and the project group had a thorough test plan that was followed and made the testing go smoothly. The test subjects managed to perform the tasks asked of them and followed another method called the "think aloud method", where every action that was made was communicated orally to the one responsible for conducting the testing. The methods used in the project has worked and gone well, but not everything is perfect in any given situation.               
\subsection{Hva fungerte ikke? Hvorfor fungerte det ikke?}
When using Agile or any other form of project management method. The team working on the project must follow the structure otherwise it won't serve it's function. The project group has done well in following Agile while using Scrum. But with fixed meetings with guidance teachers and assignment provider the "scrums" and short meetings haven't been used in a way it should. The project group also in the beginning of the project found weekdays where it was possible to meet and work together. Negating any need for short meetings, as any planning could be done at anytime. Agile software development values working software over comprehensive documentation. This contradicts what is done in the project where the project is judged more based on documentation than the software and the documentation have been prioritized at certain points over software development. Agile was however what was most suited for the project and except from the mentions above served it's purpose. 

The questionnaire that was sent out got twenty responses, and the data collected was valuable and the project got a lot of usable information from it. But the structure and the order of the questions could have been better planned. Some of the recipients answered questions seemingly thinking there wouldn't be a follow up question, and had to either repeat themselves or ignored the question. Splitting the questionnaire into two questionnaires could have been an alternative, where one was for those with a digital system already in place and one for those without. This might a have resulted in more accurate and consistent answers. Although the project group got usable data the questionnaire could have been different and didn't work as well as anticipated. (?) %vet ikke om denne delen trengs eller skal flyttes, spørreskjemaet funket forsåvidt

agile, brukt men ikke helt 

questionnaire svar men kunne vart mer strukturert etc

interview, stregnerre struyktur mer kontroll, flere

\section{Delivery}
For this project the group did set up some main and sub goals for what is going to be delivered in this project. The main goal was to make a inventory module which provides an overview of the equipment available at MakerSpace. The sub goal was to to build a lending module from what the group already had made. Alongside with the main and sub goals, it was made some smaller goals that was made so the group could reach the main and sub goals. In this section it can be read what did the group deliver, and what did not.

\subsection{What has not been achieved?}
Early on the group had the sub goal about lending included in the planning, and it was also made some database model of how a lending system would work, and it was also some discussions about what should be included in that kind of an system. Unfortunately the group started to realize when it was some trouble to get the inventory module to work as the group wanted to, that the time for building a lending module was running out. The group soon started to realize how big task it was to build a lending system is and how complex it is to develop. As that the group understand that situation, it was made a decision that the main focus was going to make a complete innovatory module as possible.

As the end of the project the group can now see that the complete inventory module that was supposed to be made, also did not go as planned. After having some big trouble getting the frontend to work as wished, it delayed the group from reaching the main goal. The group underestimated how complex it is to develop a completely complete innovatory system. After that the group has been looking at different 
solution for how to display the frontend, it was made one solution. The one solution that was made, turned out not work as it was supposed to. Because of that, the group needed to redevelop the frontend 
solution. By that a lot of time has been spent trying to display a working frontend. As the group was having a weekly meetings with both the supervisor and assignment provider, they was both aware of situation and the progression of the project. 

\subsection{What has been achieved?}
Even as the main and sub goals did not go as planned, it was still something that the group could achieve. The group have develop a solid backend with a working application programming interface (API). It has also been set up a working database structure for the inventory system. The group has also 
formulated a questionnaire that has been sent out to other Maker spaces in Norway and have been receiving good result. With the result from the questionnaire it has been collected good idea's that can be use full for further develop.

\section{Hva har vi lært? - Andreas}
- Hva har vi fått ut av prosjektet
- hvor mye jobb det faktisk er
- Hvor viktig det er å bruke 

\section{Hva gikk bra/ikke bra? - Morten??}
\subsection{Hva gikk bra?}
The group believes several tasks were completed well.

Routing ...

An evaluation of Finn.no was conducted.
The group used the knowledge gathered during the evaluation to design and shape the wireframes.
The group believes the wireframes proved useful as each individual group member knew what the final product was to emulate.
The group could thus more easily assign tasks to individual group members and have that group member know what is the expected outcome of the assignment.
The group thus believes the evaluation along with the later designed wireframes proved very useful during the development of the application. 

Singel page app(?) ...

The backend primarily supports two features; fetching items as well as authenticating users.
These two features are exposed through endpoints accessible to the frontend allowing the frontend to communicate with the backend.
The two features are currently not used together - meaning there are no items which require user authentication to fetch.
A desired quality of the application has been since the start of the project to make the application extendable.
These two features shape the core of the application and future development can thus proceed in whatever direction is deemed desirable by the future developers of the project.
The group thus believes the development of the backend went well.

Tester ...

Intervju ...

\subsection{Hva gikk ikke bra?}
Gikk ikke bra
Frontend - Vue

\subsection{Hva ville vi gjort annerledes? }
- Hatt en design student
-flere intervju(magnus added)

\todo{Mål nådd? Fungerte metodene dere beskrev?}
The project has been a journey for the group. With both ups and downs, but it has gone well despite not achieving everything the group had hoped to achieve. In chapter five the project group will discuss and voice opinions on how the project went. This includes what the group think went well, not so well, deviations, general thoughts and what the project group has learned from the project.   

\section{What went not so ducky}
Something that proved a challenge was Vue.js. Vue.js was not that easy as it was thought to be. The single page application feature of Vue.js is certainly a good feature, but the specific code and template that needed to be used proved to be more difficult to handle than anticipated and it worked in the end, but might have been easier using React instead or something that might have been more familiar to the project group to use. CSS was  also a bit problematic as it is something that no one in the group had much experience with. And was one of the more difficult things to get working. Styling and positioning combined with usability proved time consuming and one fixed error resulted in another error. Where it seemed like an endless circle of chasing small errors. But this isn't unusual for software development however it was not planned to have any problems with it. 

cookie section her....----??

During the time of writing the report the project group has used an online editor called Overleaf. When using sources it was originally thought to use APA 6th. This did not happen because the packages Overleaf has for APA 6th and the bibliography the project group used, ended with the whole document reformatting when switching to APA 6th. This reformatting ruined the entire setup both chapters, subsections, picture positions, text formatting and headlines changed drastically. Quite a lot of time was spent fixing and trying to make it work, but in the end it was decided to go back to a non APA 6th solution. 

\section{The groups thoughts}

\section{Deviations from the original plan}
For one we planned on including a loaning-module to the MMS, but the scope for the project had to be re-adjusted and the loaning-module was left to be developed by someone else in the future. So the project group could focus on the core features. The core features being the backend and it's database and the frontend with it's web application, registering new users, logging in and communication with the backend. 

CSS...?
\section{What the group would have done differently}
With the knowledge the project group has now. There is a set of things that could have been done differently. One of these things is setting the report up properly. Overleaf is a fantastic tool with many opportunities, but knowing how to properly set it up and doing it at the beginning would have saved the project group from some problems. APA 6th could have been properly setup if the project group knew doing it at a later point instead of adding a package at the start would ruin the report structure.

When the project group choice to use a full JavaScript stack, Vue.js was chosen as the frontend technology to be used. Vue.js single page application proved to be harder to work with than anticipated and if the project group were to change something, it would probably be something differently than Vue.js. It would be something the project group is more familiar with or not develop around a single page application.  

\section{What the group learned}


 
  



Describes the project process - what went well, what did not and what could have been done differently.
The chapter also outlines what the group has learned during the project.

- catogori ting relasjon

- Får lov å kjøre uten TLS

- Turns of service, legges til av oppdragsgiver(juridiske årsaker) 

- avik(ting vi ikke har fått gjort)
1. ikke laget utlånmodel
2. coockies(?)
3. 

\chapter{Conclusion}
Presents the results and outlines the continuation of the project.

\printbibliography

\chapter{Attachments}

\subsection{Heuristic evaluation results}

\begin{enumerate}
    \item \textbf{How are things categorized?}
    
\textbf{Group member:} 
The first thing that you see when going to finn.no is lots of elements that is categories. It is up to 14 different categories, and each of the elements is visualized with symbols, alongside with a category name. That is an good combination to help the user to choose the right category when they are going to look after stuff to buy. With only having symbols, it would lead up to misunderstanding, since some symbols can be too old for someone (as an old telephone for the young people), or people can have different interpretations about the symbolse. It is a good way to categorize. 

\textbf{Assignment provider:}
Things are categorized by primary type like “Eiendom”, “Bil”, “Torget” and then categories by sub category under the primary category. Some categories have rather deep category trees.

    \item \textbf{How does search work?}

\textbf{Group member:} 
The search is also something you find on the first site. In the search it have been placed a placeholder/label with a few examples to help the user to know how or what they can search for. 

When searching for example “TV”, you will get how many hits you get for each categories. This is a good thing, since if you know that you are searching for a “TV” as in a television, then you know that you need to choose the category where you can find television, and not the category about apartment. 

\textbf{Assignment provider:}
Search works very well. When in a category the search only searches in that category, which is a rather powerful feature of the system. On the main page however, search works on all items on the site.

    \item \textbf{How many things can you see on a page before switching?}

\textbf{Group member:} 
When you are looking on all the hits you got by searching television, you can see up to 51 elements before you can switch to the next site. The buy elements is organized like it is tree elements side by side, and then a new line with tree new elements. 

\textbf{Assignment provider:}
50 items in the electronics category.

    \item \textbf{Can you create your own user?}

\textbf{Group member:} 
Yes it is possible to create your own user on finn.no. In the upper right corner of the front page, it says “My FINN”. When clicking on that you will be taken to a page where you got the opportunity to logg inn or make a user. When clicking on “make a user” you will need to fill in email and password. 

\textbf{Assignment provider:}
Yes. It’s fairly simple, but you do have to verify with BankID to authenticate your account.

    \item \textbf{Is it easy to find contact information?}

\textbf{Group member:} 
To find contact information to FINN, you need to scroll down to the footer. It is written with small texts, and can for someone be hard to find at first. It should be placed on top or more visible so the user don't need to look for it. But on the other hand the FINN’s contact information is not the most important thing, it is to find the contact information to the one you are gonna buy a item from.

Without clicking the advertisement, the only information you get is where the item is and if the are a company or private seller. When you click on the advertisement you will on the right side get the name of the seller and right below it you will find a button you can click that give you opportunity to send a message/email to the seller. With other words it is easy to find.

\textbf{Assignment provider:}
Yes, it’s easy to find contact information for both Finn.no and individual sellers.
	
    \item \textbf{How does navigation work?}

\textbf{Group member:} 
When doing navigation on the main page you go through the most important things first. You start with search. After that you will navigate through all of the category that exist and then through some popular advertisement. At last you navigate through the footer where social media and contact information is. 

When you are inside of an advertisement you navigate first through the picture and information about the item. After that you going through contact information about the seller. It feel neat and orientated to navigate through FINN.

\textbf{Assignment provider:}
As Finn.no is a site for selling and buying stuff navigation primarily works on item category. As a sales site the navigation is pretty effective.

    \item \textbf{Can you store things you are interested in?}

\textbf{Group member:} 
FINN offers a element that let you store advertisement that you are interesting or want to get information about it if it has been updated or of it still for sale. Before clicking the advertisement you will see on the ad that its a heart symbol. When clicking that you will store it in your favorit. You can also add it to favorit when you have clicked on it, it will show a element below the picture that says “Add to favorit”

\textbf{Assignment provider:}
Yes. If you have an account, you just need to push the heart or “Legg til favoritt”.
	
    \item \textbf{What does the buy item contain?}

\textbf{Group member:} 
On the item it shows picture(s) of the product, follow by the title for the advertisement. For som product it also shows specific things and equipment and below that you will see the price for it. After that you can read the description about the product.

\textbf{Assignment provider:}
There are no “Buy item” button on the site. As the site is an action site you need to message the seller to be able to buy the item. Professional sellers that use Finn.no to showcase their products have a “Buy in online store” button where you can buy the items. To buy stuff from private sellers you need to send the seller a message and then buy the item directly.
\end{enumerate}

\textbf{Summary}

Since both of the evaluators is having two different expertise it was expected to get some different result of the evaluating. That is good for seeing more than one perspective of looking into FINN’s website. This is what that have been found by following the eight heuristics: 

\textbf{1.  How are things categorized?}
Things are categorized by primary type and then by sub category. They also use symbols/icons to represent the categories. 

\textbf{2.  How does search work?}
The search is placed good where it is very simple
to find it. When using it, it will in a category only searches in that category, which is a powerful feature of the system. On the main page it will search for all of the items. 

\textbf{3.  How many things can you see on a page before switching?}
50 element is possible to see. 

\textbf{4.  Can you create your own user?}
Yes it is possible and very simple. When creating a user one need to use BankID to authenticate your account.

\textbf{5.  Is it easy to find contact information?}
Yes, it is simple to find contact information for the seller, but a little tricky to find contact information for FINN. It has not been prioritized since it will be some of the last thing one will get by navigating through the main page. 

\textbf{6.  How does navigation work?}
The navigation go through the most important first, with that it means sals items and it works alright. 

\textbf{7.  Can you store things you are interested in?}
If one is having a account, then it is possible to store things of interest. It is a button that can be clicked on for storing. 

\textbf{8.  What does the buy item contain?}
It contain picture, title, description and other important information that needed before buying. For private sellers one need to take direct contact/message the seller to buy the item. For professional sellers it will be displayed a “Buy in online store” button.

\subsection{Test plan for usability testing}
\begin{enumerate}
  \item \textbf{Purpose:} This is where you describe the purpose of what you will find with this usability test.
  \item \textbf{Problem statement:} Here you need to make some kind of list of problems that you want to be answered through the test.
  \item \textbf{User profile:} This is where you need to describe what kind of users that is supposed to take the test. What age are they? What background information do they have on what they will be testing? It is important to describe your user to see if that will affect the results of the test.
  \item \textbf{Methodology:} This is more of an concrete way to describe how the test should be carried out. Often it is four steps that should be followed. Those four steps are: 
  \begin{enumerate}
  \item \textbf{Participant greeting and background questionnaire:} This is where you can greet the participants and make them feel welcome. Do a background questionnaire and make sure that an anonymity agreement is signed.
  
  \item \textbf{Orientation:} The participant will get a short, verbal and scripted introduction and orientation about the test.
  The purpose of the test will be explained and what is expected from them as a participant. The participant will also be informed that they will be observed, videotaped or audio taped.
  
  \item \textbf{Performance test:} The performance test consist of a series of tasks and/or scenarios that the participant will be asked to do while being observed. The observer will write down any of the participants behavior, comment and any other unusual circumstances that might affect the result of the test.       
  
  \item \textbf{Participant debriefing:} After the test is done, it's time for a debriefing with the participant. It should follow these steps:
  
  - Filling out some kind of a brief performance questionnaire of the usability of the system. 
  
  - Let the participant give an overall comment of her or his performance of the test.
  
  - Let the participant tell about errors or problems that occurred during the test.
  
  \end{enumerate}
  \item \textbf{Task list:} This will contain a list of tasks that the monitor of the test want the participants to do. This is in order for the monitor to be able to test the things that he or she wants to test.
  
  \item \textbf{Test environment and equipment requirements:} This one need to describe the environment that the participant will be in. Is the environment quiet and calm or is it a lot of disturbance around that can disturb and affect the test result? That is important to document.
  
  Alongside with a description of the environment, it can be necessary to write down what kind of equipment that will be needed to complete the test. An example can be if there is a need for a pen, paper or maybe a phone. 
  
  \item \textbf{Test monitor role:} This is a description of what kind of role the test monitor will have during the test. Is the monitor going to be closed or open. By that it means will the monitor be closed as in the participant wont get any help if they ask, or if the monitor will be open where the monitor can answer questions and give some leads.
  
  It will need to be describe what the test monitor will be looking for. If the test is recorded, are the monitor still gonna write notes during the test, or is he or she just gonna look at the recording afterwards?  
  
  \item \textbf{Evaluation measure:} This is what kind of information or data that will be collected through the testing. It can be how long the participant use to complete the task, how many errors that occurs or other things that is needs to be collected.

  \item \textbf{Test report contents and presentation:} This is what the usability test consist of, and what the reader can find. It mainly consists of the test plan, the result and findings found in the test.
\end{enumerate}
\end{document}
